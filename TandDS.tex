% This started out as a .org file, but it's simpler to edit the .tex (for now)
% POPL format
\documentclass[acmsmall,review,anonymous,prologue,dvipsnames]{acmart}
\settopmatter{printfolios=true,printccs=false,printacmref=false}

% We intend to use Agda. But not in this file. Still:
\usepackage[utf8]{inputenc}
\usepackage[T1]{fontenc}

\usepackage{graphicx}
\usepackage{grffile}
\usepackage{longtable}
\usepackage{wrapfig}
\usepackage{rotating}
\usepackage[normalem]{ulem}
\usepackage{amsmath,amssymb}
\usepackage{textcomp}
\usepackage{capt-of}
\usepackage{hyperref}
\usepackage{multicol}

% Might upgrade to minted later, right now, go with Agda default
% \usepackage[]{minted}
% \makeatletter
% \AtBeginEnvironment{minted}{\dontdofcolorbox}
% \def\dontdofcolorbox{\renewcommand\fcolorbox[4][]{##4}}
% \makeatother

% use biblatex -- but will want to insure we use the right ACM format
%\usepackage[backend=biber,style=alphabetic]{biblatex}
%\addbibresource{MyReferences.bib}

% POPL-related
\acmJournal{PACMPL}
\acmVolume{1}
\acmNumber{POPL}
\acmArticle{1}
\acmYear{2020}
\acmMonth{1}
\acmDOI{}
\setcopyright{none}

% Unicode - needed?
\usepackage{MyUnicodeSymbols}
\def\with{\kern0.7em \withrule \kern0.7em }
\def\withrule{\vrule height1.57ex depth0.43ex width0.12em}
\newunicodechar{❙}{\ensuremath{\mathop{\with}}}

% set up some colours
\definecolor{darkred}{rgb}{0.3, 0.0, 0.0}
\definecolor{darkgreen}{rgb}{0.0, 0.3, 0.1}
\definecolor{darkblue}{rgb}{0.0, 0.1, 0.3}
\definecolor{darkorange}{rgb}{1.0, 0.55, 0.0}
\definecolor{sienna}{rgb}{0.53, 0.18, 0.09}
\hypersetup{colorlinks,linkcolor=darkblue,citecolor=darkblue,urlcolor=darkgreen}

\date{\today}
\title{Theories and Data Structures\\\medskip
\large ---Draft---}
\hypersetup{
 pdfauthor={\href{mailto:alhassm@mcmaster.ca}{Musa Al-hassy}, \href{mailto:carette@mcmaster.ca}{Jacques Carette}, \href{mailto:kahl@cas.mcmaster.ca}{Wolfram Kahl}},
 pdftitle={Theories and Data Structures},
 pdfkeywords={},
 pdfsubject={Work done at McMaster University, 2017--2019.},
 pdflang={English}}

\begin{document}
\maketitle

% super-cheap edcomm macro. Might want to upgrade this
\def\edcomm#1#2{ \fbox{\textbf{Comment: #1 }} #2 \fbox{\textbf{End Comment}}}

\author{Musa Al-hassy}
\affiliation{
  \institution{McMaster University}
  \streetaddress{1280 Main St. W.}
  \city{Hamilton}
  \state{ON}
  \postcode{L8S 4K1}
  \country{Canada}}
\email{alhassym@mcmaster.ca}
\author{Jacques Carette}
\author{Wolfram Kahl}

\begin{abstract}


\textbf{placeholder} We give a rational reconstruction of some common (and
not-so-common) data-structures that arise in functional
programming. Our categorical approach also leads us to defining
standard functions which ought to be in all reasonable libraries of
data-structures. Being systematic in the exploration of the design
space reveals quite a lot of structure and information about
data-structures and their origins.

\end{abstract}

\section{Introduction}
\label{sec:intro}

It is relatively well-known in functional programming folklore that lists and monoids
are somewhow related. With a little prodding, most functional programmers will recall
(or reconstruct) that lists are, in fact, an instance of a monoid. But when asked if there
is a deeper relation, fewer are able to conjure up ``free
monoid''. Fewer still would be able formally prove this relation, in
other words, to actually fill in all the parts
that make up the adjunction between the forgetful functor from the category of monoids (and
monoid homomorphisms) and the category of types (and functions) and
the free monoid functor. To do so in full detail is, however, quite
informative --- and we will proceed to do so below.
\edcomm{MA}{ It is important to mention that this has been worked out in
numerous other writings. That this is not the prime novelty of the work. E.g.;
when a library claims to support X does it actually provide the necessaity
‘kit’ that that X /intersincly/ comes with?}

So as to never be able to cheat, cut corners, etc, we will do all of
our work in Agda, with this document\footnote{Sources are available on github}
being literate.  But when we do, something interesting happens: we are forced
to write some rather useful functions over lists. Somehow \texttt{map},
\texttt{\_++\_} and \texttt{fold} are all \emph{required}.

But is this somehow a fluke? Of course not! So, what happens when we
try to explore this relationship?

A programmer's instinct might be to start poking around various
data-structures to try and see which also give rise to a similar
relation. This is a rather difficult task: not all of them arise this
way. Instead, we start from the opposite end: systematically write
down ``simple'' theories, and look at what pops out of the
requirements of having a ``left adjoint to the forgetful
functor''. This turns out to be very fruitful, and the approach we
will take here.

Naturally, we are far from the first to look at this. 
\edcomm{JC}{ Fill in the related work here. From Universal Algebra through to many papers of Hinze}. In other words, the \emph{theory}
behind what we'll be talking about here is well known.

So why bother? Because, in practice, there is just as much beauty in
the details as there was in the theory! By \emph{systematically} going
through simple theories, we will create a dictionary between theories
and a host of useful data-structures. Many of which do not in fact
exist in the standard libraries of common (and uncommon) functional
languages. And even when they do exist, all the ``kit'' that is derived
from the theory is not uniformly provided.

Along the way, we meet several roadblocks, some of which are rather
surprising, as results from the (theory) literature tell us that there
really ought to be no problems there. Only when we dig deeper do we
understand what is going on: classical mathematics is not
constructive! So even when type theorists were busy translating
results for use in functional programming, by not actually proving
their results in a purely constructive meta-theory, they did not
notice these roadblocks.
Surmounting these problems will highlight how
different axioms, via their \emph{shape}, will naturally give rise to
data-structures easily implementable with inductive types, and which
require much more machinery.

In short, our contributions:
\begin{itemize}
\item a systematic exploration of the space of simple theories
\item giving a complete dictionary
\item highlighting the ``kit'' that arises from fully deriving all the
adjunctions
\item a survey of which languages' standard library offers what structures
(and what kit)
\end{itemize}

\section{ Monoids and lists}
\label{sec:monoid}

\edcomm{JC}{Give the full details}

\section{Exploring simple theories}
\label{sec:exploring}
\edcomm{JC}\{Not fully sure how to go about this, while staying
leisurely\}

\section{Trouble in paradise}
\label{sec:trouble}
Commutative Monoid, idempotence, and so on.

\section{Survey of implementations}

\section{Conclusion}

\label{sec:survey}

\end{document}
