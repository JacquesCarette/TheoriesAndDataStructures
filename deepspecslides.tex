%
% Slides to be used by Musa to present at DeepSpec2018 in Princeton.
%
% Some of the slides are dense but this is because I only have 5 minutes!
%

%% Tentative abstract.
%%
%% Simple mathematical theories give rise to common data-structures such as
%% natural numbers, lists, trees, and nullables. Why and how do these structures
%% arise? We argue that they are the result of minimal constructions of associated
%% theories. Moreover the usual toolkit of combinators associated with a data-structure
%% is “built into” it. We point out that the average programmer *thinks* in the
%% category of classical Sets but actually *works* in the category Types, consisting
%% of types with computable functions, which lacks quotients and hence certain
%% “expected” constructions cannot be realised. We argue that bags, among others,
%% are not a construction in Types and hence not as accessible to programmers
%% as other inductive datatypes.

\documentclass[serif,mathserif,10pt]{beamer}
\usepackage{ucs}
\usepackage[utf8x]{inputenc}
\usepackage{etex}
\usepackage{graphicx}
\usepackage{textgreek}

\usepackage{multicol}

% \usepackage{pgf}
\usepackage{tikz}

\usepackage{amssymb}
\usepackage{hyperref}
\usepackage{picture,xcolor}
\def\Id{\mathsf{Id}}

\usepackage{color}
\definecolor{grey}{gray}{0.6}
\definecolor{slidered}{rgb}{1,0,0}
\definecolor{slidegreen}{rgb}{0,1,0}
\definecolor{slideblue}{rgb}{0,0,1}
\definecolor{slidepurple}{rgb}{0,1,1}
\definecolor{slidegray}{rgb}{0.5,0.5,0.5}
\definecolor{slideivory}{rgb}{1,1,0.9}
\definecolor{slideblueb}{rgb}{0,0.5,0.8}
\newcommand{\highlight}[1]{\textcolor{slidered}{#1}}
\newcommand{\sred}[1]{\textcolor{slidered}{#1}}
\newcommand{\sblue}[1]{\textcolor{slideblue}{#1}}
\newcommand{\sblueb}[1]{\textcolor{slideblueb}{#1}}
\newcommand{\sgreen}[1]{\textcolor{slidegreen}{#1}}
\newcommand{\sgray}[1]{\textcolor{slidegray}{#1}}
\newcommand{\spurple}[1]{\textcolor{purple}{#1}}

\beamertemplatenavigationsymbolsempty
\usetheme{Boadilla}
\usecolortheme{beaver}

\DeclareUnicodeCharacter{7472}{\ensuremath{^\mathsf{D}}} % MODIFIER LETTER CAPITAL D
\DeclareUnicodeCharacter{9723}{\ensuremath{\square}} % WHITE MEDIUM SQUARE
\DeclareUnicodeCharacter{119920}{\ensuremath{\mathbf{I}}} % MATHEMATICAL BOLD ITALIC CAPITAL I
\DeclareUnicodeCharacter{119991}{\ensuremath{_\mathsf{B}}} % MATHEMATICAL SCRIPT SMALL B
\DeclareUnicodeCharacter{120008}{\ensuremath{_\mathsf{S}}} % MATHEMATICAL SCRIPT SMALL S
\DeclareUnicodeCharacter{120001}{\ensuremath{_\mathsf{L}}} % MATHEMATICAL SCRIPT SMALL L
\DeclareUnicodeCharacter{119997}{\ensuremath{_\mathsf{H}}} % MATHEMATICAL SCRIPT SMALL H
\DeclareUnicodeCharacter{8348}{\ensuremath{_\mathsf{t}}} % MATHEMATICAL SCRIPT SMALL T
\DeclareUnicodeCharacter{8339}{\ensuremath{_\mathsf{x}}} % MATHEMATICAL SCRIPT SMALL X
\DeclareUnicodeCharacter{119925}{\ensuremath{\mathcal{N}}} % MATHEMATICAL BOLD ITALIC CAPITAL N
\DeclareUnicodeCharacter{119924}{\ensuremath{\mathcal{M}}} % MATHEMATICAL BOLD ITALIC CAPITAL M
\DeclareUnicodeCharacter{7484}{\ensuremath{\mathsf{O}}} % MODIFIER LETTER CAPITAL O
\DeclareUnicodeCharacter{9678}{\ensuremath{\mathsf{BULLSEYE}}} % BULLSEYE
\DeclareUnicodeCharacter{8667}{\ensuremath{\Rrightarrow}} % rightwards triple arrow
\DeclareUnicodeCharacter{8255}{\ensuremath{\smile}} % undertie, subscript-converse
\DeclareUnicodeCharacter{8265}{\ensuremath{! \! ? }} % exclamation question mark
\DeclareUnicodeCharacter{9632}{\ensuremath{ \square }} % black square
\DeclareUnicodeCharacter{9679}{\ensuremath{ \boldsymbol{\cdot} }} % black circle
\DeclareUnicodeCharacter{9675}{\ensuremath{ \boldsymbol{\circ} }} % white circle

% \title{A tale of theories and data-structures}
% \title{Computing Science is Adjoint to Mathematics}
\title{The Treasure's in the Details}
\author[Carette, Al-hassy, Kahl]{Jacques Carette, Musa Al-hassy, Wolfram Kahl}
\institute[McMaster]{McMaster University, Hamilton}
% 

\newtheorem{claim}{Claim}


% for adding a forced line break inside a table cell; e.g., \thead{line1 \\ line2}
\usepackage{makecell}

\begin{document}

% ====================================================================================


\frame{
  \titlepage
  
  \centerline{\color{blue} \url{https://github.com/JacquesCarette/TheoriesAndDataStructures}}
  \centerline{\sc work in progress}
  \vspace{2em}
  \centerline{\underline{\color{white}{.\hspace{20em}.} }}
  \centerline{DeepSpec 2018, Princeton University}
}

\begin{frame} \frametitle{A Relationship Between Mathematics and Computing Science}

  \begin{block}{ The Curry-Howard Correspondence \hfill{\small ---``Propositions as Types''} }
    \begin{tabular}{cll}
      \textbf{Logic} & \textbf{Programming} & Example Use in Programming \\ \hline
      % \thead{proof \\ p is a proof of P } & \thead{typing \\ p is of type P} & hi \\ \hline
      
      $true$ & singleton type
           &{\footnotesize return type of side-effect only methods} \\
                            % similar to C's \texttt{void} \\

      $false$ & empty type
           &{\footnotesize return type for non-terminating methods } \\ \hline\pause

      ⇒ & function type \hfill →
         &{\footnotesize methods with an input and output type } \\

      ∧ & product type \hfill ×
         &{\footnotesize simple records of data and methods } \\

      ∨ & sum type \hfill +
         &{\footnotesize enumerations or tagged unions} \\ \hline\pause
      
      ∀ & dependent function type \hfill Π
         & {\footnotesize return type varies according to input \emph{value} } \\
      
      ∃ & dependent product type \hfill Σ
         &{\footnotesize record fields depend on each other's \emph{values} } \\ \hline\pause
      
      \thead{natural \\ deduction} & type system
           &{\footnotesize ensuring only ``meaningful'' programs } \\

      {\footnotesize hypothesis} & free variables
           &{\footnotesize global variables, closures } \\

      \thead{implication \\ elimination} & function application
           &{\footnotesize executing methods on arguments } \\
      
      \thead{implication \\ introduction} & λ-abstraction
      & \thead{\hspace{-3.5em}parameters acting as local variables
        \\ \hspace{-9.5em}to method definitions } \\
      
    \end{tabular}
  \end{block}

% X-axis to the right and Y-axis upwards
\begin{picture}(0,0)(0,0)
  \pause
  \put(0,225){\LARGE \spurple{\textbf{\emph{\fbox{Proposition}} $P$ is either $true$ or $false$.}}}
  
  \pause
  \put(0,40){\LARGE \spurple{\textbf{\emph{\fbox{Type}} $P$ may have more than 2 elements!}}}
  
  \pause
  \put(90,210){\rotatebox{-45}{\huge \sred{\textbf{Not An Isomorphism!}}}}
\end{picture}
  
\end{frame}

\begin{frame} \frametitle{Adjunctions, à la \texttt{Prop} }
  
  \begin{enumerate}
  \item Try to understand \emph{complex notion} via a transformation
    to a \emph{simpler notion}. \pause
  \item Methodology: Transport $complex$-problems into $simple$-setting where they are
    \alert{easier to solve}, then go back. \pause
  \item Usually transformation is not invertible, but has a \alert{best approximate inverse}: \pause
  \end{enumerate}  
  
  \begin{block}{Galois Connections}
    \[ f \,⊣\, g \qquad≡\qquad ∀x,y \;\bullet\; f\, x ≤₁ y  \;≡\; x ≤₂ g\, y\] 
  \end{block}

  \pause
  For example, we can understand the complicated notion of supremum
  % ---the least upper bound, what a mouthful!---
  by the trivial notion of constant function:
  % {\footnotesize --then showing $\sup(h \sqcup k) = \sup\, h \sqcup \sup k$ is easy!}
  \[ ∀ h, z \;\bullet\; \sup\, h ≤ z \;≡\; h \overset{.}{≤} \mathsf{K}\, z \] 

  \begin{picture}(0,0)(0,0)
    \pause
    \put(120,225){\footnotesize Easy now to show that supremum distributes over maximum}
    \put(170,213){\bf \sblue{$\sup(h \overset{.}{\sqcup} k) = \sup\, h \sqcup \sup k$}}
  \end{picture}
  %
  \pause
  Ubiquitous! {\footnotesize Integer division: $(× k) ⊣ (÷ n)$,
  floor: $\iota_\mathbb{R} ⊣ \lfloor-\rfloor$,
  \\ Prefix extraction: $\mathsf{length} × \mathsf{Id} ⊣ \mathsf{take}$,
  Binary joins: $\max ⊣ \Delta$, Power sets: $\bigcup ⊢ \mathbb{P}$,
  \\ Unit \& empty types, residuals, colimits, Kan extensions,
  \alert{Free ⊢ Forgetful}  }
  
\end{frame}

\begin{frame} \frametitle{Adjunctions, Constructively}

  \alert{Proof relevance:} We care about \emph{which proof} witnesses the Galois Connection. \pause
  We make a few replacements, \pause
  \begin{tabular}{lcl}
    Order `≤'    &   $\quad↦\quad$ & Morphism type constructor `⟶' \\ \pause
    Equivalence `≡' & $\quad↦\quad$ & Isomorphism `≅' \\ \pause
    Functions $f,g$ between sets   & $\quad↦\quad$ & Functors $F,G$ between categories \pause
  \end{tabular}
  
  \begin{block}{Adjunctions, Global Characterisation}
    \[ F \,⊣\, G \qquad≡\qquad
    ∀x,y \;\bullet\; (F\, x ⟶₁ y) \;≅\; (x ⟶₂ G\, y)\]
    Where the isomorphism is natural in $x$ and $y$.
  \end{block} \pause
  
  \begin{block}{Adjunctions, Local Characterisation}
    % counit ε, unit η
    \centering $\epsilon : FG \overset{.}{⟶} \Id
    \qquad \text{ and } \qquad \eta : \Id \overset{.}{⟶} GF$
    \\ such that the nonsensical “inverse equation $\Id = \epsilon ∘ \eta$” holds.

      {\footnotesize
        $\eta$ is called the \alert{unit} since $FG$ is a monad;
        dually $\epsilon$ is the \alert{counit} since $GF$ is a comonad. }
  \end{block} \pause
  
  $\eta$ embeds, or \emph{inserts}, “elements” \emph{as} “singleton structures”; \\
  whereas $\epsilon$ \emph{extracts} “elements” \emph{from} “singleton structures.”
  
\end{frame}

\begin{frame} \frametitle{Computing Science $\;\;\dashv\;\;$ Mathematics}
  Given an arbitrary type $A$, \\ \vspace*{0mm}
  \pause
  \begin{block}{}
  \begin{tabular}{llll}
    \textbf{Theory} & \textbf{Structure} & Over & Equality \\ \hline
    Carrier & Identity A & Type  & Propositional \\
    Pointed & Maybe A & Type & Propositional\\ \hline \pause
    Unary & ℕ × A &  Type & Propositional\\
    Involutive & A ⊎ A & Type & Propositional \\ \hline\pause
    Magma & Tree A &  Type & Propositional\\
    Semigroup & NEList A & Type & Propositional\\ \hline\pause
    Monoid & List A & Type & Propositional\\
    Left Unital Semigroup & List A × ℕ & Type & Propositional\\
    Right Unital Semigroup & ℕ × List A & Type & Propositional\\ \hline\pause
    Commutative Monoid & Bag & Setoid &  Proof-relevant permutations \\
    Group & ? & ? & ? \\
    Abelian Group & Hybrid Sets & Setoid &  Proof-relevant permutations\\
    Idemp. Comm. Monoid & Set & Setoid &  Logical equivalence \\
  \end{tabular}
  \end{block}

  {\footnotesize
    \pause
    $\diamond$ \sblue{Free Structure} is ``the''
  \sred{\textbf{term language in normal form}} associated to the theory.

  \pause
  $\diamond \; \bot$, $\top$, $\mathbb{B}$, $\mathbb{N}$, $\mathbb{Z}$ show up
  as \emph{initial objects}.
  }
  
  % X-axis to the right and Y-axis upwards
  \setlength{\unitlength}{1cm}
  \begin{picture}(0,0)(0,0)      
   \pause
   \put(3,3.3){\rotatebox{45}{\huge \sblue{\textbf{Easy! $\sim\!$Three Months!}}}}
  
   \pause
   \put(3.6,2.8){\LARGE \spurple{\textbf{\emph{⇐ This took one year}}!}}
   
   \pause 
   \put(-1, -0.4){\colorbox{white}{ \framebox(30,3){}}}
   \put(0.2,2.3){ Implementation issues!  }
   \pause \put(0.5, 1.9){ $\diamond$ Inductive type ⇒ Ordering! Thus not free! }
   \pause \put(0.5,1.5){ $\diamond \; A → ℕ$ \hspace{3em} ⇒ No finite support }
   \pause \put(0.5,1.1){
     \hspace{7.2em} ⇒ Finiteness hard to express constructively!  }
   \pause \put(0.5,0.7){
     \hspace{7.2em} ⇒ Decidable equality on $A$! Thus not free! }
   \pause \put(0.5,0.3){ $\diamond \; \mathsf{List}\, A$ with permutations! \sred{Works!} }

   \pause
   \put(0.5, 4.8){\colorbox{yellow}{ \framebox(11,1.1){}}}
   \put(0.8, 5.5){ Theorem :: Within Martin-Löf Type Theory, there's no \emph{free} functor}
   \put(0.8, 5.0){  from \emph{Types} to \emph{Commutative Monoids} using propositional equality.}
   
  \end{picture}

  \iffalse
  + Implementation issues!
     - inductive type ⇒ ordering!
     - A → ℕ   ⇒ no finite support; finiteness hard to express constructively!
                ⇒ decidble equality on A
     - `List A` with permutations!


  \fi

  
\end{frame}

\begin{frame} \frametitle{Why the Mechanisation \hfill
    {\footnotesize $\sim1000$ lines of Agda code}}

{\footnotesize

  \pause
  \vspace{-1em}
  \begin{block}{Benefits of the formal approach}
    \begin{itemize} \itemsep0em      
    \item Give solid grounding to folklore; e.g., there are no associative
      extensions of non-associative operators: There is no free functor from magmas to semigroups.

    \pause
    \item Unicode and mixfix operators ⇒ Mechanisation is readable and not significantly more effort than a conventional presentation in LaTeX.

    \pause
    \item Agda as a type checker for doing mathematics —manipulating symbols according to specified rules. Mechanised mathematical notation ⇒ \alert{Confidence in results!} Dispell silly conjectures/errors!

    \pause      
    \item Agda enables a natural treatment of theories and their direct use as modules of executable programs.
      % \item formalisations can be used both for theoretical reasoning and for executable implementations

    \pause      
    \item Finally, formal proofs are fool-proof! \textbf{No “an exercise to the reader”!}      
    \end{itemize}
  \end{block}
  
  \pause
  \vspace{-0.8em}
  \def\map{\mathsf{map}\,}  
  \begin{block}{By working out the details we \alert{discovered} }
    \begin{itemize} \itemsep0em
    \item Common recursion principles % for trees, lists, nullables, ….
      ; “Evaluate” the \alert{syntax} under a given \alert{semantics}.

    \pause      
    \item \texttt{map}~operators~and~their~optimisation~rewrites:~$\map \Id \mapsto \Id \; ; \; \map f \circ \map g \mapsto \map (f ∘ g)$

    \pause      
    \item (Haskell's \texttt{return}) Unit transformations take on familiar forms:
      \texttt{[-], Leaf, Just,} $…$
      Along with proven optimisation laws:
      $\map f (\mathsf{return}\, x) \mapsto \mathsf{return}\, (f\, x)$
      
      \pause      
    \item \alert{Boom Hierarchy does not necessarily work for type theory!}

    \pause      
    \item New-ish relationships: Lists may be specified as a \emph{free monoid}
      but may also be specified as \emph{initial pointed indexed unary algebra}.
      % roughly put, “a minimal semiautomaton”.
      Single-sorted unary algebras also have a surprising relationship with temporal logic. %: The \texttt{Eventually A}.
    \end{itemize}
  \end{block}

}
  
\end{frame}

\begin{frame} \frametitle{Adventurous Avenues To Explore
    \hfill {\footnotesize ---Where do these show up?}}

  \pause
  \vspace{-1.4em}
  \begin{block}{Math theories yielding potential data structures}
    {\small left-zero monoid \sred{(Prolog's cut!?)}, pointed unary \sred{(natural numbers!?)}, idempotent unary, commutative
magma, pointed magma, quasigroup, loop, semilattice, medial magma,
left semimedial magma, left distributive magma, idempotent magma,
zeropotent magma, left unary magma, Steiner magma, null semigroup, BCI algebra, BCK
algebra, squag, sloop, Moufang quasigroup, loop, left shelf, shelf,
rack, spindle, quandle, Kei, involutive semigroup, band, rectangular
band, hemigroup, pseudo inverse algebra, ringoid, left near semiring,
near semiring, semifield, semiring, semirng, pre-dioid, dioid, star semiring,
idempotent dioid, ring, commutative ring, idempotent semiring, Stone algebra,
Kleene lattice, Kleene algebra, Heyting algebra, Goedel algebra, ortho
lattice, directoid, semiheap, idempotent semiheap, heap, meadow, wheel. }
\end{block}

  \pause
  \begin{block}{Structures possibly arising from mathematical theories}
        Difference list \sred{(Yoneda on Monoids!?)}, stack, queue, finite map, rose tree, digraph,
multigraph, partitions, oriented cycles, colorings, tri-colorings,
hedges, derangements, ballots, commutative parenthesizations, linear
order, permutations, even permutations, chains, oriented sets, even
sets, octopus, vertebrae, automata \sred{(pointed indexed algebras!?)}.
  \end{block}  
\end{frame}

\iffalse
\begin{frame}
\frametitle{Lists and Monoids}
\begin{claim}
A \sred{List} is a \sred{Free Monoid}
\end{claim}
What does that really mean? \\

\pause
\vspace*{4mm}
\sred{Fancy explanation}: The functor from the category \textsf{Types} of \sgreen{types and function}, with
\texttt{List} as its object mapping and \texttt{map} for homomorphism,
to the category \textsf{Monoid} of \sgreen{monoids and monoid homomorphisms}, is \sblue{left
adjoint} to the forgetful functor (from \textsf{Monoid} to \textsf{Types}).

\pause
\vspace*{4mm}
\texttt{List} (equipped with constructors \texttt{[]}, \texttt{::} and
functions \texttt{map}, \texttt{++},
\texttt{singleton}, and \texttt{foldr}) is the
\sblue{language of monoids}. In other words, \texttt{List} 
is the canonical term syntax for 
\sblue{computing with monoids}.

\pause
\vspace*{4mm}
Why on earth would we care about that? Let's see!
\end{frame}
\fi

\end{document}

% Local Variables:
% eval: (setq NAMEtex (buffer-name) NAME (file-name-sans-extension (buffer-name)))
% eval: (defun makepdf () (shell-command (concat "pdflatex " NAMEtex)))
% eval: (defun showpdf () (shell-command (concat "evince " NAME ".pdf &")))
% compile-command: (progn (save-buffer) (makepdf) (makepdf))
% End:
