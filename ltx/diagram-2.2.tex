% DIAGRAM MACROS (version 2.2)

% If a document has been previously realized using version 1.0 or 1.1
% of the DIAGRAM macros and if you have to recompile it using now
% version 2.2, it is compulsory to modify in this document each
% occurence of \Diag, \Diagv, \diaG, \diagsV and \move
% according to the new rules.

% Typeset the document DIAGRAM.READ.ME to get user's information
% about the following macros.


% MACROS FOR DRAWING IN-TEXT PICTURES

% \tlowername{P}{f} puts the name f under the picture P
\newcommand{\tlowername}[2]%
{$\stackrel{\makebox[1pt]{#1}}%
{\begin{picture}(0,0)%
\put(0,0){\makebox(0,6)[t]{\makebox[1pt]{$#2$}}}%
\end{picture}}$}%

% \tcase{P} draws the picture P with length 20pt
\newcommand{\tcase}[1]{\makebox[23pt]%
{\raisebox{2.5pt}{#1{20}}}}%

% \Tcase{P}{f} draws the picture P with upper name f
% and length 20pt.
\newcommand{\Tcase}[2]{\makebox[23pt]%
{\raisebox{2.5pt}{$\stackrel{#2}{#1{20}}$}}}%

% \tbicase{P} draws the bi-picture P with length 20pt
\newcommand{\tbicase}[1]{\makebox[23pt]%
{\raisebox{1pt}{#1{20}}}}%

% \Tbicase{P}{f}{g} draws the bi-picture P with names f, g
% and length 20pt.
\newcommand{\Tbicase}[3]{\makebox[23pt]{\raisebox{-7pt}%
{$\stackrel{#2}{\mbox{\tlowername{#1{20}}{\scriptstyle{#3}}}}$}}}%




% IN-TEXT ARROWS

% \AR{n} draws an arrow of length n units
\newcommand{\AR}[1]%
{\begin{picture}(#1,0)%
\put(0,0){\vector(1,0){#1}}%
\end{picture}}%

% \DOTAR{n} draws a dotted  arrow of length n units
\newcommand{\DOTAR}[1]%
{\NUMBEROFDOTS=#1%
\divide\NUMBEROFDOTS by 3%
\begin{picture}(#1,0)%
\multiput(0,0)(3,0){\NUMBEROFDOTS}{\circle*{1}}%
\put(#1,0){\vector(1,0){0}}%
\end{picture}}%


% \MONO{n} draws a monomorphism of length n units
\newcommand{\MONO}[1]%
{\begin{picture}(#1,0)%
\put(0,0){\vector(1,0){#1}}%
\put(2,-2){\line(0,1){4}}%
\end{picture}}%

% \EPI{n} draws an epimorphism of length n units
\newcommand{\EPI}[1]%
{\begin{picture}(#1,0)(-#1,0)%
\put(-#1,0){\vector(1,0){#1}}%
\put(-6,-2){\line(0,1){4}}%
\end{picture}}%

% \BIMO{n} draws a bimorphism of length n units
\newcommand{\BIMO}[1]%
{\begin{picture}(#1,0)(-#1,0)%
\put(-#1,0){\vector(1,0){#1}}%
\put(-6,-2){\line(0,1){4}}%
\put(-#1,-2){\hspace{2pt}\line(0,1){4}}%
\end{picture}}%

% \BIAR{n} draws a pair of arrows of length n units
\newcommand{\BIAR}[1]%
{\begin{picture}(#1,4)%
\put(0,0){\vector(1,0){#1}}%
\put(0,4){\vector(1,0){#1}}%
\end{picture}}%

% \EQL{n} draws an equality of length n units
\newcommand{\EQL}[1]%
{\begin{picture}(#1,0)%
\put(0,1){\line(1,0){#1}}%
\put(0,-1){\line(1,0){#1}}%
\end{picture}}%

% \ADJAR{n} draws a pair of adjoint arrows of length n units
\newcommand{\ADJAR}[1]%
{\begin{picture}(#1,4)%
\put(0,0){\vector(1,0){#1}}%
\put(#1,4){\vector(-1,0){#1}}%
\end{picture}}%


% All the following cammands produce arrows with length 20pt
% between 1.5pt spaces.

% arrow
\newcommand{\ar}{\tcase{\AR}}%

% arrow with upper name [1]
\newcommand{\Ar}[1]{\Tcase{\AR}{#1}}%

% dotted arrow
\newcommand{\dotar}{\tcase{\DOTAR}}%

% dotted arrow with upper name [1]
\newcommand{\Dotar}[1]{\Tcase{\DOTAR}{#1}}%


% monomorphism
\def\mono{\tcase{\MONO}}%

% monomorphism with upper name [1]
\newcommand{\Mono}[1]{\Tcase{\MONO}{#1}}%

% epimorphism
\newcommand{\epi}{\tcase{\EPI}}%

% epimorphism with upper name [1]
\newcommand{\Epi}[1]{\Tcase{\EPI}{#1}}%

% bimorphism
\newcommand{\bimo}{\tcase{\BIMO}}%

% bimorphism with upper name [1]
\newcommand{\Bimo}[1]{\Tcase{\BIMO}{#1}}%

% isomorphism
\newcommand{\iso}{\Tcase{\AR}{\cong}}%

% isomorphism with upper name [1]
\newcommand{\Iso}[1]{\Tcase{\AR}{\cong{#1}}}%

% pair of arrows
\newcommand{\biar}{\tbicase{\BIAR}}%

% pair of arrows with names [1],[2]
\newcommand{\Biar}[2]{\Tbicase{\BIAR}{#1}{#2}}%

% equality
\newcommand{\eql}{\tcase{\EQL}}%

% pair of adjoint arrows
\newcommand{\adjar}{\tbicase{\ADJAR}}%

% pair of adjoint arrows with names [1],[2]
\newcommand{\Adjar}[2]{\Tbicase{\ADJAR}{#1}{#2}}%




% IN-TEXT BACK ARROWS

% \BKAR{n} draws a pointing back arrow of length n units
\newcommand{\BKAR}[1]%
{\begin{picture}(#1,0)%
\put(#1,0){\vector(-1,0){#1}}%
\end{picture}}%

% \BKDOTAR{n} draws a backward dotted  arrow
% of length n units
\newcommand{\BKDOTAR}[1]%
{\NUMBEROFDOTS=#1%
\divide\NUMBEROFDOTS by 3%
\begin{picture}(#1,0)%
\multiput(#1,0)(-3,0){\NUMBEROFDOTS}{\circle*{1}}%
\put(0,0){\vector(-1,0){0}}%
\end{picture}}%


% \BKMONO{n} draws a pointing back monomorphism of length n units
\newcommand{\BKMONO}[1]%
{\begin{picture}(#1,0)(-#1,0)%
\put(0,0){\vector(-1,0){#1}}%
\put(-2,-2){\line(0,1){4}}%
\end{picture}}%

% \BKEPI{n} draws a pointing back epimorphism of length n units
\newcommand{\BKEPI}[1]%
{\begin{picture}(#1,0)%
\put(#1,0){\vector(-1,0){#1}}%
\put(6,-2){\line(0,1){4}}%
\end{picture}}%

% \BKBIMO{n} draws a pointing back bimorphism of length n units
\newcommand{\BKBIMO}[1]%
{\begin{picture}(#1,0)%
\put(#1,0){\vector(-1,0){#1}}%
\put(6,-2){\line(0,1){4}}%
\put(#1,-2){\hspace{-2pt}\line(0,1){4}}%
\end{picture}}%

% \BKBIAR{n} draws a pair of pointing back arrows of length n units
\newcommand{\BKBIAR}[1]%
{\begin{picture}(#1,4)%
\put(#1,0){\vector(-1,0){#1}}%
\put(#1,4){\vector(-1,0){#1}}%
\end{picture}}%

% \BKADJAR{n} draws a pair of adjoint arrows of length n units
\newcommand{\BKADJAR}[1]%
{\begin{picture}(#1,4)%
\put(0,4){\vector(1,0){#1}}%
\put(#1,0){\vector(-1,0){#1}}%
\end{picture}}%



% All the following commands produce back arrows with length 20pt
% between 1.5pt spaces.

% back arrow
\newcommand{\bkar}{\tcase{\BKAR}}%

% back arrow with upper name [1]
\newcommand{\Bkar}[1]{\Tcase{\BKAR}{#1}}%

% backward dotted arrow
\newcommand{\bkdotar}{\tcase{\BKDOTAR}}%

% backward dotted arrow with upper name [1]
\newcommand{\Bkdotar}[1]{\Tcase{\BKDOTAR}{#1}}%

% back monomorphism
\newcommand{\bkmono}{\tcase{\BKMONO}}%

% back monomorphism with upper name [1]
\newcommand{\Bkmono}[1]{\Tcase{\BKMONO}{#1}}%

% back epimorphism
\newcommand{\bkepi}{\tcase{\BKEPI}}%

% back epimorphism with upper name [1]
\newcommand{\Bkepi}[1]{\Tcase{\BKEPI}{#1}}%

% back bimorphism
\newcommand{\bkbimo}{\tcase{\BKBIMO}}%

% back bimorphism with upper name [1]
\newcommand{\Bkbimo}[1]{\Tcase{\BKBIMO}{\hspace{9pt}#1}}%

% back isomorphism
\newcommand{\bkiso}{\Tcase{\BKAR}{\cong}}%

% back isomorphism with upper name [1]
\newcommand{\Bkiso}[1]{\Tcase{\BKAR}{\cong{#1}}}%

% pair of back arrows
\newcommand{\bkbiar}{\tbicase{\BKBIAR}}%

% pair of back arrows with names [1],[2]
\newcommand{\Bkbiar}[2]{\Tbicase{\BKBIAR}{#1}{#2}}%

% back equality
\newcommand{\bkeql}{\tcase{\EQL}}%

% back pair of adjoint arrows
\newcommand{\bkadjar}{\tbicase{\BKADJAR}}%

% back pair of adjoint arrows with names [1],[2]
\newcommand{\Bkadjar}[2]{\Tbicase{\BKADJAR}{#1}{#2}}%




% MACROS FOR DRAWING HORIZONTAL PICTURES

% \lowername{P}{f} puts the name f under the picture P
\newcommand{\lowername}[2]%
{$\stackrel{\makebox[1pt]{#1}}%
{\begin{picture}(0,0)%
\truex{600}%
%\put(0,0){\makebox(0,\value{x})[t]{\makebox[1pt]{$#2$}}}% Changed by W. Kahl
\put(0,-600){\makebox(0,\value{x})[t]{\makebox[1pt]{$#2$}}}%
\end{picture}}$}%

% \hcase{P}{n} draws the picture P with length n units
\newcommand{\hcase}[2]%
{\makebox[0pt]%
{\raisebox{-1pt}[0pt][0pt]{#1{#2}}}}%

% \Hcase{P}{f}{n} draws the picture P with upper name f
% and length n units.
\newcommand{\Hcase}[3]%
{\makebox[0pt]
{\raisebox{-1pt}[0pt][0pt]%
{$\stackrel{\makebox[0pt]{$\textstyle{#2}$}}{#1{#3}}$}}}%

% \hcasE{P}{f}{n} draws the picture P with lower name f
% and length n units.
\newcommand{\hcasE}[3]%
{\makebox[0pt]%
%{\raisebox{-9pt}[0pt][0pt]% changed by W. Kahl
{\raisebox{-2pt}[0pt][0pt]%
{\lowername{#1{#3}}{#2}}}}%

% \hbicase{P}{n} draws the bi-picture P with length n units.
\newcommand{\hbicase}[2]%
{\makebox[0pt]%
{\raisebox{-2.5pt}[0pt][0pt]{#1{#2}}}}%

% \Hbicase{P}{f}{g}{n} draws the bi-picture P with names f, g
% and length n units.
\newcommand{\Hbicase}[4]%
{\makebox[0pt]
{\raisebox{-10.5pt}[0pt][0pt]%
{$\stackrel{\makebox[0pt]{$\textstyle{#2}$}}%
{\mbox{\lowername{#1{#4}}{#3}}}$}}}%




% EAST ARROWS

% \EAR{n} draws an east arrow of length n units
\newcommand{\EAR}[1]%
{\begin{picture}(#1,0)%
\put(0,0){\vector(1,0){#1}}%
\end{picture}}%

% \EDOTAR draws a dotted arrow of length n units
\newcommand{\EDOTAR}[1]%
{\truex{100}\truey{300}%
\NUMBEROFDOTS=#1%
\divide\NUMBEROFDOTS by \value{y}%
\begin{picture}(#1,0)%
\multiput(0,0)(\value{y},0){\NUMBEROFDOTS}%
{\circle*{\value{x}}}%
\put(#1,0){\vector(1,0){0}}%
\end{picture}}%


% \EMONO{n} draws an east monomorphism of length n units
\newcommand{\EMONO}[1]%
{\begin{picture}(#1,0)%
\put(0,0){\vector(1,0){#1}}%
\truex{300}\truey{600}%
\put(\value{x},-\value{x}){\line(0,1){\value{y}}}%
\end{picture}}%

% \EEPI{n} draws an east epimorphism of length n units
\newcommand{\EEPI}[1]%
{\begin{picture}(#1,0)(-#1,0)%
\put(-#1,0){\vector(1,0){#1}}%
\truex{300}\truey{600}\truez{800}%
\put(-\value{z},-\value{x}){\line(0,1){\value{y}}}%
\end{picture}}%

% \EBIMO{n} draws an east bimorphism of length n units
\newcommand{\EBIMO}[1]%
{\begin{picture}(#1,0)(-#1,0)%
\put(-#1,0){\vector(1,0){#1}}%
\truex{300}\truey{600}\truez{800}%
\put(-\value{z},-\value{x}){\line(0,1){\value{y}}}%
\put(-#1,-\value{x}){\hspace{3pt}\line(0,1){\value{y}}}%
\end{picture}}%

% \EBIAR{n} draws an east pair of arrows of length n units
\newcommand{\EBIAR}[1]%
{\truex{400}%
\begin{picture}(#1,\value{x})%
\put(0,0){\vector(1,0){#1}}%
\put(0,\value{x}){\vector(1,0){#1}}%
\end{picture}}%

% \EEQL{n} draws an east equality of length n units
\newcommand{\EEQL}[1]%
{\begin{picture}(#1,0)%
\truex{200}%
\put(0,\value{x}){\line(1,0){#1}}%
\put(0,0){\line(1,0){#1}}%
\end{picture}}%

% \EADJAR{n} draws an east pair of adjoint arrows of length n
% units
\newcommand{\EADJAR}[1]%
{\truex{400}%
\begin{picture}(#1,\value{x})%
\put(0,0){\vector(1,0){#1}}%
\put(#1,\value{x}){\vector(-1,0){#1}}%
\end{picture}}%


% All the following commands produce east arrows
% raised at the adequate position and of formal dimensions (0,0)

% east arrow of length [1]x100 units
\newcommand{\earv}[1]{\hcase{\EAR}{#100}}%

% east arrow of adjusted length
\newcommand{\ear}%
{\hspace{\SOURCE\unitlength}%
\hcase{\EAR}{\ARROWLENGTH}}%

% east arrow with upper name [1] and length [2]x100 units
\newcommand{\Earv}[2]{\Hcase{\EAR}{#1}{#200}}%

% east arrow with upper name [1] and adjusted length
\newcommand{\Ear}[1]%
{\hspace{\SOURCE\unitlength}%
\Hcase{\EAR}{#1}{\ARROWLENGTH}}%

% east arrow with lower name [1] and length [2]x100 units
\newcommand{\eaRv}[2]{\hcasE{\EAR}{#1}{#200}}%

% east arrow with lower name [1] and adjusted length
\newcommand{\eaR}[1]%
{\hspace{\SOURCE\unitlength}%
\hcasE{\EAR}{#1}{\ARROWLENGTH}}%

% east dotted arrow of length [1]x100 units
\newcommand{\edotarv}[1]{\hcase{\EDOTAR}{#100}}%

% east dotted arrow of adjusted length
\newcommand{\edotar}%
{\hspace{\SOURCE\unitlength}%
\hcase{\EDOTAR}{\ARROWLENGTH}}%

% east dotted arrow with upper name [1] and length [2]x100 units
\newcommand{\Edotarv}[2]{\Hcase{\EDOTAR}{#1}{#200}}%

% east dotted arrow with upper name [1] andadjusted length
\newcommand{\Edotar}[1]%
{\hspace{\SOURCE\unitlength}%
\Hcase{\EDOTAR}{#1}{\ARROWLENGTH}}%

% east dotted arrow with lower name [1] and length [2]x100 units
\newcommand{\edotaRv}[2]{\hcasE{\EDOTAR}{#1}{#200}}%

% east dotted arrow with lower name [1] and adjusted length
\newcommand{\edotaR}[1]%
{\hspace{\SOURCE\unitlength}%
\hcasE{\EDOTAR}{#1}{\ARROWLENGTH}}%

% east monomorphism with length [1]x100 units
\newcommand{\emonov}[1]{\hcase{\EMONO}{#100}}%

% east monomorphism with adjusted length
\newcommand{\emono}%
{\hspace{\SOURCE\unitlength}%
\hcase{\EMONO}{\ARROWLENGTH}}%

% east monomorphism with upper name [1] and length [2]x100 units
\newcommand{\Emonov}[2]{\Hcase{\EMONO}{#1}{#200}}%

% east monomorphism with upper name [1] and adjusted length
\newcommand{\Emono}[1]%
{\hspace{\SOURCE\unitlength}%
\Hcase{\EMONO}{#1}{\ARROWLENGTH}}%

% east monomorphism with lower name [1] and length [2]x100 units
\newcommand{\emonOv}[2]{\hcasE{\EMONO}{#1}{#200}}%

% east monomorphism with lower name [1] and adjusted length
\newcommand{\emonO}[1]%
{\hspace{\SOURCE\unitlength}%
\hcasE{\EMONO}{#1}{\ARROWLENGTH}}%

% east epimorphism with length [1]x100 units
\newcommand{\eepiv}[1]{\hcase{\EEPI}{#100}}%

% east epimorphism with adjusted length
\newcommand{\eepi}%
{\hspace{\SOURCE\unitlength}%
\hcase{\EEPI}{\ARROWLENGTH}}%

% east epimorphism with upper name [1] and length [2]x100 units
\newcommand{\Eepiv}[2]{\Hcase{\EEPI}{#1}{#200}}%

% east epimorphism with upper name [1] and adjusted length
\newcommand{\Eepi}[1]%
{\hspace{\SOURCE\unitlength}%
\Hcase{\EEPI}{#1}{\ARROWLENGTH}}%

% east epimorphism with lower name [1] and length [2]x100 units
\newcommand{\eepIv}[2]{\hcasE{\EEPI}{#1}{#200}}%

% east epimorphism with lower name [1] and adjusted length
\newcommand{\eepI}[1]%
{\hspace{\SOURCE\unitlength}%
\hcasE{\EEPI}{#1}{\ARROWLENGTH}}%

% east bimorphism with length [1]x100 units
\newcommand{\ebimov}[1]{\hcase{\EBIMO}{#100}}%

% east bimorphism with adjusted length
\newcommand{\ebimo}%
{\hspace{\SOURCE\unitlength}%
\hcase{\EBIMO}{\ARROWLENGTH}}%

% east bimorphism with upper name [1] and length [2]x100 units
\newcommand{\Ebimov}[2]{\Hcase{\EBIMO}{#1}{#200}}%

% east bimorphism with upper name [1] and adjusted length
\newcommand{\Ebimo}[1]%
{\hspace{\SOURCE\unitlength}%
\Hcase{\EBIMO}{#1}{\ARROWLENGTH}}%

% east bimorphism with lower name [1] and length [2]x100 units
\newcommand{\ebimOv}[2]{\hcasE{\EBIMO}{#1}{#200}}%

% east bimorphism with lower name [1] and adjusted length
\newcommand{\ebimO}[1]%
{\hspace{\SOURCE\unitlength}%
\hcasE{\EBIMO}{#1}{\ARROWLENGTH}}%

% east isomorphism with length [1]x100 units
\newcommand{\eisov}[1]{\Hcase{\EAR}{\cong}{#100}}%

% east isomorphism with adjusted length
\newcommand{\eiso}%
{\hspace{\SOURCE\unitlength}%
\Hcase{\EAR}{\cong}{\ARROWLENGTH}}%

% east isomorphism with upper name [1] and length [2]x100 units
\newcommand{\Eisov}[2]{\Hcase{\EAR}{\cong#1}{#200}}%

% east isomorphism with upper name [1] and adjusted length
\newcommand{\Eiso}[1]%
{\hspace{\SOURCE\unitlength}%
\Hcase{\EAR}{\cong#1}{\ARROWLENGTH}}%

% east isomorphism with lower name [1] and length [2]x100 units
\newcommand{\eisOv}[2]{\hcasE{\EAR}{\cong#1}{#200}}%

% east isomorphism with lower name [1] and adjusted length
\newcommand{\eisO}[1]%
{\hspace{\SOURCE\unitlength}%
\hcasE{\EAR}{\cong#1}{\ARROWLENGTH}}%

% pair of east arrows of length [1]x100 units
\newcommand{\ebiarv}[1]{\hbicase{\EBIAR}{#100}}%

% pair of east arrows of adjusted length
\newcommand{\ebiar}%
{\hspace{\SOURCE\unitlength}%
\hbicase{\EBIAR}{\ARROWLENGTH}}%

% pair of east arrows with names [1] [2] and length [3]x100 units
\newcommand{\Ebiarv}[3]{\Hbicase{\EBIAR}{#1}{#2}{#300}}%

% pair of east arrows with names [1] [2] and adjusted length
\newcommand{\Ebiar}[2]%
{\hspace{\SOURCE\unitlength}%
\Hbicase{\EBIAR}{#1}{#2}{\ARROWLENGTH}}%

% east equality of length [1]x100 units
\newcommand{\eeqlv}[1]{\hcase{\EEQL}{#100}}%

% east equality of adjusted length
\newcommand{\eeql}%
{\hspace{\SOURCE\unitlength}%
\hbicase{\EEQL}{\ARROWLENGTH}}%

% east pair of adjoint arrows of length [1]x100 units
\newcommand{\eadjarv}[1]{\hbicase{\EADJAR}{#100}}%

% east pair of adjoint arrows of adjusted length
\newcommand{\eadjar}%
{\hspace{\SOURCE\unitlength}%
\hbicase{\EADJAR}{\ARROWLENGTH}}%

% east pair of adjoint arrows with names [1][2] and length [3]x100units
\newcommand{\Eadjarv}[3]{\Hbicase{\EADJAR}{#1}{#2}{#300}}%

% east pair of adjoint arrows with names [1]J[2] and adjusted length
\newcommand{\Eadjar}[2]%
{\hspace{\SOURCE\unitlength}%
\Hbicase{\EADJAR}{#1}{#2}{\ARROWLENGTH}}%




% WEST ARROWS

% \WAR{n} draws a pointing back arrow of length n units
\newcommand{\WAR}[1]%
{\begin{picture}(#1,0)%
\put(#1,0){\vector(-1,0){#1}}%
\end{picture}}%

% \WDOTAR draws a dotted arrow of length n units
\newcommand{\WDOTAR}[1]%
{\truex{100}\truey{300}%
\NUMBEROFDOTS=#1%
\divide\NUMBEROFDOTS by \value{y}%
\begin{picture}(#1,0)%
\multiput(#1,0)(-\value{y},0){\NUMBEROFDOTS}%
{\circle*{\value{x}}}%
\put(0,0){\vector(-1,0){0}}%
\end{picture}}%

% \WMONO{n} draws a pointing back monomorphism of length n units
\newcommand{\WMONO}[1]%
{\begin{picture}(#1,0)(-#1,0)%
\put(0,0){\vector(-1,0){#1}}%
\truex{300}\truey{600}%
\put(-\value{x},-\value{x}){\line(0,1){\value{y}}}%
\end{picture}}%

% \WEPI{n} draws a pointing back epimorphism of length n units
\newcommand{\WEPI}[1]%
{\begin{picture}(#1,0)%
\put(#1,0){\vector(-1,0){#1}}%
\truex{300}\truey{600}\truez{800}%
\put(\value{z},-\value{x}){\line(0,1){\value{y}}}%
\end{picture}}%

% \WBIMO{n} draws a pointing back bimorphism of length n units
\newcommand{\WBIMO}[1]%
{\begin{picture}(#1,0)%
\put(#1,0){\vector(-1,0){#1}}%
\truex{300}\truey{600}\truez{800}%
\put(\value{z},-\value{x}){\line(0,1){\value{y}}}%
\put(#1,-\value{x}){\hspace{-3pt}\line(0,1){\value{y}}}%
\end{picture}}%

% \WBIAR{n} draws a pair of pointing back arrows of length n units
\newcommand{\WBIAR}[1]%
{\truex{400}%
\begin{picture}(#1,\value{x})%
\put(#1,0){\vector(-1,0){#1}}%
\put(#1,\value{x}){\vector(-1,0){#1}}%
\end{picture}}%

% \WADJAR{n} draws a pair of adjoint arrows of length n units
\newcommand{\WADJAR}[1]%
{\truex{400}%
\begin{picture}(#1,\value{x})%
\put(0,\value{x}){\vector(1,0){#1}}%
\put(#1,0){\vector(-1,0){#1}}%
\end{picture}}%


% All the following commands produce west arrows
% raised at the adequate position and of formal dimensions (0,0)

%  west arrow of length [1]x100 units
\newcommand{\warv}[1]{\hcase{\WAR}{#100}}%

%  west arrow of adjusted length
\newcommand{\war}%
{\hspace{\SOURCE\unitlength}%
\hcase{\WAR}{\ARROWLENGTH}}%

% west arrow with upper name [1] and length [2]x100 units
\newcommand{\Warv}[2]{\Hcase{\WAR}{#1}{#200}}%

% west arrow with upper name [1] and adjusted length
\newcommand{\War}[1]%
{\hspace{\SOURCE\unitlength}%
\Hcase{\WAR}{#1}{\ARROWLENGTH}}%

% west arrow with lower name [1] and length [2]x100 units
\newcommand{\waRv}[2]{\hcasE{\WAR}{#1}{#200}}%

% west arrow with lower name [1] and adjusted length
\newcommand{\waR}[1]%
{\hspace{\SOURCE\unitlength}%
\hcasE{\WAR}{#1}{\ARROWLENGTH}}%

% west dotted arrow of length [1]x100 units
\newcommand{\wdotarv}[1]{\hcase{\WDOTAR}{#100}}%

% west dotted arrow of adjusted length
\newcommand{\wdotar}%
{\hspace{\SOURCE\unitlength}%
\hcase{\WDOTAR}{\ARROWLENGTH}}%

% west dotted arrow with upper name [1] and length [2]x100 units
\newcommand{\Wdotarv}[2]{\Hcase{\WDOTAR}{#1}{#200}}%

% west dotted arrow with upper name [1] andadjusted length
\newcommand{\Wdotar}[1]%
{\hspace{\SOURCE\unitlength}%
\Hcase{\WDOTAR}{#1}{\ARROWLENGTH}}%

% west dotted arrow with lower name [1] and length [2]x100 units
\newcommand{\wdotaRv}[2]{\hcasE{\WDOTAR}{#1}{#200}}%

% west dotted arrow with lower name [1] and adjusted length
\newcommand{\wdotaR}[1]%
{\hspace{\SOURCE\unitlength}%
\hcasE{\WDOTAR}{#1}{\ARROWLENGTH}}%

% west monomorphism with length [1]x100 units
\newcommand{\wmonov}[1]{\hcase{\WMONO}{#100}}%

% west monomorphism with adjusted length
\newcommand{\wmono}%
{\hspace{\SOURCE\unitlength}%
\hcase{\WMONO}{\ARROWLENGTH}}%

% west monomorphism with upper name [1] and length [2]x100 units
\newcommand{\Wmonov}[2]{\Hcase{\WMONO}{#1}{#200}}%

% west monomorphism with upper name [1] and adjusted length
\newcommand{\Wmono}[1]%
{\hspace{\SOURCE\unitlength}%
\Hcase{\WMONO}{#1}{\ARROWLENGTH}}%

% west monomorphism with lower name [1] and length [2]x100 units
\newcommand{\wmonOv}[2]{\hcasE{\WMONO}{#1}{#200}}%

% west monomorphism with lower name [1] and adjusted length
\newcommand{\wmonO}[1]%
{\hspace{\SOURCE\unitlength}%
\hcasE{\WMONO}{#1}{\ARROWLENGTH}}%

% west epimorphism with length [1]x100 units
\newcommand{\wepiv}[1]{\hcase{\WEPI}{#100}}%

% west epimorphism with adjusted length
\newcommand{\wepi}%
{\hspace{\SOURCE\unitlength}%
\hcase{\WEPI}{\ARROWLENGTH}}%

% west epimorphism with upper name [1] and length [2]x100 units
\newcommand{\Wepiv}[2]{\Hcase{\WEPI}{#1}{#200}}%

% west epimorphism with upper name [1] and adjusted length
\newcommand{\Wepi}[1]%
{\hspace{\SOURCE\unitlength}%
\Hcase{\WEPI}{#1}{\ARROWLENGTH}}%

% west epimorphism with lower name [1] and length [2]x100 units
\newcommand{\wepIv}[2]{\hcasE{\WEPI}{#1}{#200}}%

% west epimorphism with lower name [1] and adjusted length
\newcommand{\wepI}[1]%
{\hspace{\SOURCE\unitlength}%
\hcasE{\WEPI}{#1}{\ARROWLENGTH}}%

% west bimorphism with length [1]x100 units
\newcommand{\wbimov}[1]{\hcase{\WBIMO}{#100}}%

% west bimorphism with adjusted length
\newcommand{\wbimo}%
{\hspace{\SOURCE\unitlength}%
\hcase{\WBIMO}{\ARROWLENGTH}}%

% west bimorphism with upper name [1] and length [2]x100 units
\newcommand{\Wbimov}[2]{\Hcase{\WBIMO}{#1}{#200}}%

% west bimorphism with upper name [1] and adjusted length
\newcommand{\Wbimo}[1]%
{\hspace{\SOURCE\unitlength}%
\Hcase{\WBIMO}{#1}{\ARROWLENGTH}}%

% west bimorphism with lower name [1] and length [2]x100 units
\newcommand{\wbimOv}[2]{\hcasE{\WBIMO}{#1}{#200}}%

% west bimorphism with lower name [1] and adjusted length
\newcommand{\wbimO}[1]%
{\hspace{\SOURCE\unitlength}%
\hcasE{\WBIMO}{#1}{\ARROWLENGTH}}%

% west isomorphism with length [1]x100 units
\newcommand{\wisov}[1]{\Hcase{\WAR}{\cong}{#100}}%

% west isomorphism with adjusted length
\newcommand{\wiso}%
{\hspace{\SOURCE\unitlength}%
\Hcase{\WAR}{\cong}{\ARROWLENGTH}}%

% west isomorphism with upper name [1] and length [2]x100 units
\newcommand{\Wisov}[2]{\Hcase{\WAR}{\cong#1}{#200}}%

% west isomorphism with upper name [1] and adjusted length
\newcommand{\Wiso}[1]%
{\hspace{\SOURCE\unitlength}%
\Hcase{\WAR}{#1}{\ARROWLENGTH}}%

% west isomorphism with lower name [1] and length [2]x100 units
\newcommand{\wisOv}[2]{\hcasE{\WAR}{\cong#1}{#200}}%

% west isomorphism with lower name [1] and adjusted length
\newcommand{\wisO}[1]%
{\hspace{\SOURCE\unitlength}%
\hcasE{\WAR}{#1}{\ARROWLENGTH}}%

% pair of west arrows of length [1]x100 units
\newcommand{\wbiarv}[1]{\hbicase{\WBIAR}{#100}}%

% pair of west arrows of adjusted length
\newcommand{\wbiar}%
{\hspace{\SOURCE\unitlength}%
\hbicase{\WBIAR}{\ARROWLENGTH}}%

% pair of west arrows with names [1] [2] and length [3]x100 units
\newcommand{\Wbiarv}[3]{\Hbicase{\WBIAR}{#1}{#2}{#300}}%

% pair of west arrows with names [1] [2] and adjusted length
\newcommand{\Wbiar}[2]%
{\hspace{\SOURCE\unitlength}%
\Hbicase{\WBIAR}{#1}{#2}{\ARROWLENGTH}}%

% west equality of length [1]x100 units
\newcommand{\weqlv}[1]{\hbicase{\EEQL}{#100}}%

% west equality of adjusted length
\newcommand{\weql}%
{\hspace{\SOURCE\unitlength}%
\hbicase{\EEQL}{\ARROWLENGTH}}%

% west pair of adjoint arrows of length [1]x100 units
\newcommand{\wadjarv}[1]{\hbicase{\WADJAR}{#100}}%

% west pair of adjoint arrows of adjusted length
\newcommand{\wadjar}%
{\hspace{\SOURCE\unitlength}%
\hbicase{\WADJAR}{\ARROWLENGTH}}%

% west pair of adjoint arrows with names [1][2] and length [3]x100units
\newcommand{\Wadjarv}[3]{\Hbicase{\WADJAR}{#1}{#2}{#300}}%

% west pair of adjoint arrows with names [1]J[2] and adjusted length
\newcommand{\Wadjar}[2]%
{\hspace{\SOURCE\unitlength}%
\Hbicase{\WADJAR}{#1}{#2}{\ARROWLENGTH}}%





% MACROS FOR DRAWING VERTICAL PICTURES

% \vcase{P}{n} draws the vertical picture P with length n units.
\newcommand{\vcase}[2]{#1{#2}}%

% \Vcase{P}{f}{n} draws the vertical picture P
% with left name f and length n units.
\newcommand{\Vcase}[3]{\makebox[0pt]%
{\makebox[0pt][r]{\raisebox{0pt}[0pt][0pt]{${#2}\hspace{2pt}$}}}#1{#3}}%

% \vcasE{P}{f}{n} draws the vertical picture P
% with right name f and length n units.
\newcommand{\vcasE}[3]{\makebox[0pt]%
{#1{#3}\makebox[0pt][l]{\raisebox{0pt}[0pt][0pt]{\hspace{2pt}$#2$}}}}%

% \vbicase{P}{n} draws the vertical bi-picture P with length n units.
\newcommand{\vbicase}[2]{\makebox[0pt]{{#1{#2}}}}%

% \Vbicase{P}{f}{g}{n} draws the vertical bi-picture P
% with names f, g and length n units.
\newcommand{\Vbicase}[4]{\makebox[0pt]%
{\makebox[0pt][r]{\raisebox{0pt}[0pt][0pt]{$#2$\hspace{4pt}}}#1{#4}%
\makebox[0pt][l]{\raisebox{0pt}[0pt][0pt]{\hspace{5pt}$#3$}}}}%





% SOUTH ARROWS

% \SAR{n} draws a south arrow of length n units
% and centers it in a box of width  0pt and height 0pt
\newcommand{\SAR}[1]%
{\begin{picture}(0,0)%
\put(0,0){\makebox(0,0)%
{\begin{picture}(0,#1)%
\put(0,#1){\vector(0,-1){#1}}%
\end{picture}}}\end{picture}}%

% \SDOTAR{n} draws a south dotted arrow of length n units
% and centers it in a box of width 0pt and height 0pt
\newcommand{\SDOTAR}[1]%
{\truex{100}\truey{300}%
\NUMBEROFDOTS=#1%
\divide\NUMBEROFDOTS by \value{y}%
\begin{picture}(0,0)%
\put(0,0){\makebox(0,0)%
{\begin{picture}(0,#1)%
\multiput(0,#1)(0,-\value{y}){\NUMBEROFDOTS}%
{\circle*{\value{x}}}%
\put(0,0){\vector(0,-1){0}}%
\end{picture}}}\end{picture}}%

% \SMONO{n} draws a south monomorphism of length n units
% and centers it in a box of width 0pt and height 0pt
\newcommand{\SMONO}[1]%
{\begin{picture}(0,0)%
\put(0,0){\makebox(0,0)%
{\begin{picture}(0,#1)%
\put(0,#1){\vector(0,-1){#1}}%
\truex{300}\truey{600}%
\put(0,#1){\begin{picture}(0,0)%
\put(-\value{x},-\value{x}){\line(1,0){\value{y}}}\end{picture}}%
\end{picture}}}\end{picture}}%


% \SEPI{n} draws a south epimorphism of length n units
% and centers it in a box of width 0pt and height 0pt
\newcommand{\SEPI}[1]%
{\begin{picture}(0,0)%
\put(0,0){\makebox(0,0)%
{\begin{picture}(0,#1)%
\put(0,#1){\vector(0,-1){#1}}%
\truex{300}\truey{600}\truez{800}%
\put(-\value{x},\value{z}){\line(1,0){\value{y}}}%
\end{picture}}}\end{picture}}%

% \SBIMO{n} draws a south bimorphism of length n units
% and centers it in a box of width 0pt and height 0pt
\newcommand{\SBIMO}[1]%
{\begin{picture}(0,0)%
\put(0,0){\makebox(0,0)%
{\begin{picture}(0,#1)%
\put(0,#1){\vector(0,-1){#1}}%
\truex{300}\truey{600}\truez{800}%
\put(0,#1){\begin{picture}(0,0)%
\put(-\value{x},-\value{x}){\line(1,0){\value{y}}}\end{picture}}%
\put(-\value{x},\value{z}){\line(1,0){\value{y}}}%
\end{picture}}}\end{picture}}%

% \SBIAR{n} draws a pair of south arrows of length n units
% and centers it in a box of width 0pt and height 0pt
\newcommand{\SBIAR}[1]%
{\begin{picture}(0,0)%
\truex{200}%
\put(0,0){\makebox(0,0)%
{\begin{picture}(0,#1)\put(-\value{x},#1){\vector(0,-1){#1}}%
\put(\value{x},#1){\vector(0,-1){#1}}%
\end{picture}}}\end{picture}}%

% \SEQL{n} draws a vertical equality of length n units
% and centers it in a box of width 0pt and height 0pt
\newcommand{\SEQL}[1]%
{\begin{picture}(0,0)%
\truex{100}%
\put(0,0){\makebox(0,0)%
{\begin{picture}(0,#1)\put(-\value{x},#1){\line(0,-1){#1}}%
\put(\value{x},#1){\line(0,-1){#1}}%
\end{picture}}}\end{picture}}%

% \SADJAR{n} draws a pair of vertical adjoint arrows of length n units
% and centers it in a box of width 0pt and height 0pt
\newcommand{\SADJAR}[1]{\begin{picture}(0,0)%
\truex{200}%
\put(0,0){\makebox(0,0)%
{\begin{picture}(0,#1)\put(-\value{x},#1){\vector(0,-1){#1}}%
\put(\value{x},0){\vector(0,1){#1}}%
\end{picture}}}\end{picture}}%


% All the following commands produce south arrows
% centered in a box of width O pt and height 0pt

% south arrow of length [1]x100 units
\newcommand{\sarv}[1]{\vcase{\SAR}{#100}}%

% south arrow of length 5000 units
\newcommand{\sar}{\sarv{50}}%

% south arrow with left name [1] and length [2]x100 units
\newcommand{\Sarv}[2]{\Vcase{\SAR}{#1}{#200}}%

% south arrow with left name [1] and length 5000 units
\newcommand{\Sar}[1]{\Sarv{#1}{50}}%

% south arrow with right name [1] and length [2]x100 units
\newcommand{\saRv}[2]{\vcasE{\SAR}{#1}{#200}}%

% south arrow with right name [1] and length 5000 units
\newcommand{\saR}[1]{\saRv{#1}{50}}%

% south dotted arrow of length [1]x100 units
\newcommand{\sdotarv}[1]{\vcase{\SDOTAR}{#100}}%

% south dotted arrow of length 5000 units
\newcommand{\sdotar}{\sdotarv{50}}%

% south dotted arrow with left name [1] and length [2]x100 units
\newcommand{\Sdotarv}[2]{\Vcase{\SDOTAR}{#1}{#200}}%

% south dotted arrow with left name [1] and length 5000 units
\newcommand{\Sdotar}[1]{\Sdotarv{#1}{50}}%

% south dotted arrow with right name [1] and length [2]x100 units
\newcommand{\sdotaRv}[2]{\vcasE{\SDOTAR}{#1}{#200}}%

% south dotted arrow with right name [1] and length 5000 units
\newcommand{\sdotaR}[1]{\sdotaRv{#1}{50}}%

% south monomorphism of length [1]x100 units
\newcommand{\smonov}[1]{\vcase{\SMONO}{#100}}%

% south monomorphism of length 5000 units
\newcommand{\smono}{\smonov{50}}%

% south monomorphism with left name [1] and length [2]x100 units
\newcommand{\Smonov}[2]{\Vcase{\SMONO}{#1}{#200}}%

% south monomorphism with left name [1] and length 5000 units
\newcommand{\Smono}[1]{\Smonov{#1}{50}}%

% south monomorphism with right name [1] and length [2]x100 units
\newcommand{\smonOv}[2]{\vcasE{\SMONO}{#1}{#200}}%

% south monomorphism with right name [1] and length 5000 units
\newcommand{\smonO}[1]{\smonOv{#1}{50}}%

% south epimorphism of length [1]x100 units
\newcommand{\sepiv}[1]{\vcase{\SEPI}{#100}}%

% south epimorphism of length 5000 units
\newcommand{\sepi}{\sepiv{50}}%

% south epimorphism with left name [1] and length [2]x100 units
\newcommand{\Sepiv}[2]{\Vcase{\SEPI}{#1}{#200}}%

% south epimorphism with left name [1] and length 5000 units
\newcommand{\Sepi}[1]{\Sepiv{#1}{50}}%

% south epimorphism with right name [1] and length [2]x100 units
\newcommand{\sepIv}[2]{\vcasE{\SEPI}{#1}{#200}}%

% south epimorphism with right name [1] and length 5000 units
\newcommand{\sepI}[1]{\sepIv{#1}{50}}%

% south bimorphism of length [1]x100 units
\newcommand{\sbimov}[1]{\vcase{\SBIMO}{#100}}%

% south bimorphism of length 5000 units
\newcommand{\sbimo}{\sbimov{50}}%

% south bimorphism with left name [1] and length [2]x100 units
\newcommand{\Sbimov}[2]{\Vcase{\SBIMO}{#1}{#200}}%

% south bimorphism with left name [1] and length 5000 units
\newcommand{\Sbimo}[1]{\Sbimov{#1}{50}}%

% south bimorphism with right name [1] and length [2]x100 units
\newcommand{\sbimOv}[2]{\vcasE{\SBIMO}{#1}{#200}}%

% south bimorphism with right name [1] and length 5000 units
\newcommand{\sbimO}[1]{\sbimOv{#1}{50}}%

% south isomorphism of length [1]x100 units
\newcommand{\sisov}[1]{\vcasE{\SAR}{\cong}{#100}}%

% south isomorphism of length 5000 units
\newcommand{\siso}{\sisov{50}}%

% south isomorphism with left name [1] and length [2]x100 units
\newcommand{\Sisov}[2]%
{\Vbicase{\SAR}{#1\hspace{-2pt}}{\hspace{-2pt}\cong}{#200}}%

% south isomorphism with left name [1] and length 5000 units
\newcommand{\Siso}[1]{\Sisov{#1}{50}}%

% pair of south arrows of length [1]x100 units
\newcommand{\sbiarv}[1]{\vbicase{\SBIAR}{#100}}%

% pair of south arrows of length 5000 units
\newcommand{\sbiar}{\sbiarv{50}}%

% pair of south arrows with names [1],[2] and length [3]x100 units
\newcommand{\Sbiarv}[3]{\Vbicase{\SBIAR}{#1}{#2}{#300}}%

% pair of south arrows with names [1],[2] and length 5000 units
\newcommand{\Sbiar}[2]{\Sbiarv{#1}{#2}{50}}%

% south equality of length [1]x100 units
\newcommand{\seqlv}[1]{\vbicase{\SEQL}{#100}}%

% south equality of length 50 pt
\newcommand{\seql}{\seqlv{50}}%

% south pair of adjoint arrows of length [1]x100 units
\newcommand{\sadjarv}[1]{\vbicase{\SADJAR}{#100}}%

% south pair of adjoint arrows of length 5000 units
\newcommand{\sadjar}{\sadjarv{50}}%

% south pair of adjoint arrows with names [1],[2] and length [3]x100 units
\newcommand{\Sadjarv}[3]{\Vbicase{\SADJAR}{#1}{#2}{#300}}%

% south pair of adjoint arrows with names [1],[2] and length 5000 units
\newcommand{\Sadjar}[2]{\Sadjarv{#1}{#2}{50}}%




% NORTH ARROWS

% \NAR{n} draws a north arrow of length n pt
% and centers it in a box of width 0pt and height 0pt
\newcommand{\NAR}[1]%
{\begin{picture}(0,0)%
\put(0,0){\makebox(0,0)%
{\begin{picture}(0,#1)\put(0,0){\vector(0,1){#1}}%
\end{picture}}}\end{picture}}%

% \NDOTAR{n} draws a north dotted arrow of length n units
% and centers it in a box of width 0pt and height 0pt
\newcommand{\NDOTAR}[1]%
{\truex{100}\truey{300}%
\NUMBEROFDOTS=#1%
\divide\NUMBEROFDOTS by \value{y}%
\begin{picture}(0,0)%
\put(0,0){\makebox(0,0)%
{\begin{picture}(0,#1)%
\multiput(0,0)(0,\value{y}){\NUMBEROFDOTS}%
{\circle*{\value{x}}}%
\put(0,#1){\vector(0,1){0}}%
\end{picture}}}\end{picture}}%


% \NMONO{n} draws a north monomorphism of length n pt
% and centers it in a box of width 0pt and height 0pt
\newcommand{\NMONO}[1]%
{\begin{picture}(0,0)%
\put(0,0){\makebox(0,0)%
{\begin{picture}(0,#1)%
\put(0,0){\vector(0,1){#1}}%
\truex{300}\truey{600}%
\put(-\value{x},\value{x}){\line(1,0){\value{y}}}%
\end{picture}}}%
\end{picture}}%

% \NEPI{n} draws a north epimorphism of length n pt
% and centers it in a box of width 0pt and height 0pt
\newcommand{\NEPI}[1]%
{\begin{picture}(0,0)%
\put(0,0){\makebox(0,0)%
{\begin{picture}(0,#1)%
\put(0,0){\vector(0,1){#1}}%
\truex{300}\truey{600}\truez{800}%
\put(0,#1){\begin{picture}(0,0)%
\put(-\value{x},-\value{z}){\line(1,0){\value{y}}}\end{picture}}%
\end{picture}}}\end{picture}}%

% \NBIMO{n} draws a north bimorphism of length n pt
% and centers it in a box of width 0pt and height 0pt
\newcommand{\NBIMO}[1]%
{\begin{picture}(0,0)%
\put(0,0){\makebox(0,0)%
{\begin{picture}(0,#1)%
\put(0,0){\vector(0,1){#1}}%
\truex{300}\truey{600}\truez{800}%
\put(-\value{x},\value{x}){\line(1,0){\value{y}}}%
\put(0,#1){\begin{picture}(0,0)%
\put(-\value{x},-\value{z}){\line(1,0){\value{y}}}\end{picture}}%
\end{picture}}}\end{picture}}%

% \NBIAR{n} draws a pair of north arrows of length n pt
% and centers it in a box of width 0pt and height 0pt
\newcommand{\NBIAR}[1]%
{\begin{picture}(0,0)%
\truex{200}%
\put(0,0){\makebox(0,0)%
{\begin{picture}(0,#1)\put(-\value{x},0){\vector(0,1){#1}}%
\put(\value{x},0){\vector(0,1){#1}}%
\end{picture}}}\end{picture}}%

% \NADJAR{n} draws a pair of vertical adjoint arrows of length n pt
% and centers it in a box of width 0pt and height 0pt
\newcommand{\NADJAR}[1]{\begin{picture}(0,0)%
\truex{200}%
\put(0,0){\makebox(0,0)%
{\begin{picture}(0,#1)\put(\value{x},#1){\vector(0,-1){#1}}%
\put(-\value{x},0){\vector(0,1){#1}}%
\end{picture}}}\end{picture}}%


% All the following commands produce north arrows
% centered in a box of width O pt and height 0 pt

% north arrow of length [1]x100 units
\newcommand{\narv}[1]{\vcase{\NAR}{#100}}%

% north arrow of length 5000 units
\newcommand{\nar}{\narv{50}}%

% north arrow with left name [1] and length [2]x100 units
\newcommand{\Narv}[2]{\Vcase{\NAR}{#1}{#200}}%

% north arrow with left name [1] and length 5000 units
\newcommand{\Nar}[1]{\Narv{#1}{50}}%

% north arrow with right name [1] and length [2]x100 units
\newcommand{\naRv}[2]{\vcasE{\NAR}{#1}{#200}}%

% north arrow with right name [1] and length 5000 units
\newcommand{\naR}[1]{\naRv{#1}{50}}%

% north dotted arrow of length [1]x100 units
\newcommand{\ndotarv}[1]{\vcase{\NDOTAR}{#100}}%

% north dotted arrow of length 5000 units
\newcommand{\ndotar}{\ndotarv{50}}%

% north dotted arrow with left name [1] and length [2]x100 units
\newcommand{\Ndotarv}[2]{\Vcase{\NDOTAR}{#1}{#200}}%

% north dotted arrow with left name [1] and length 5000 units
\newcommand{\Ndotar}[1]{\Ndotarv{#1}{50}}%

% north dotted arrow with right name [1] and length [2]x100 units
\newcommand{\ndotaRv}[2]{\vcasE{\NDOTAR}{#1}{#200}}%

% north dotted arrow with right name [1] and length 5000 units
\newcommand{\ndotaR}[1]{\ndotaRv{#1}{50}}%

% north monomorphism of length [1]x100 units
\newcommand{\nmonov}[1]{\vcase{\NMONO}{#100}}%

% north monomorphism of length 5000 units
\newcommand{\nmono}{\nmonov{50}}%

% north monomorphism with left name [1] and length [2]x100 units
\newcommand{\Nmonov}[2]{\Vcase{\NMONO}{#1}{#200}}%

% north monomorphism with left name [1] and length 5000 units
\newcommand{\Nmono}[1]{\Nmonov{#1}{50}}%

% north monomorphism with right name [1] and length [2]x100 units
\newcommand{\nmonOv}[2]{\vcasE{\NMONO}{#1}{#200}}%

% north monomorphism with right name [1] and length 5000 units
\newcommand{\nmonO}[1]{\nmonOv{#1}{50}}%

% north epimorphism of length [1]x100 units
\newcommand{\nepiv}[1]{\vcase{\NEPI}{#100}}%

% north epimorphism of length 5000 units
\newcommand{\nepi}{\nepiv{50}}%

% north epimorphism with left name [1] and length [2]x100 units
\newcommand{\Nepiv}[2]{\Vcase{\NEPI}{#1}{#200}}%

% north epimorphism with left name [1] and length 5000 units
\newcommand{\Nepi}[1]{\Nepiv{#1}{50}}%

% north epimorphism with right name [1] and length [2]x100 units
\newcommand{\nepIv}[2]{\vcasE{\NEPI}{#1}{#200}}%

% north epimorphism with right name [1] and length 5000 units
\newcommand{\nepI}[1]{\nepIv{#1}{50}}%

% north bimorphism of length [1]x100 units
\newcommand{\nbimov}[1]{\vcase{\NBIMO}{#100}}%

% north bimorphism of length 5000 units
\newcommand{\nbimo}{\nbimov{50}}%

% north bimorphism with left name [1] and length [2]x100 units
\newcommand{\Nbimov}[2]{\Vcase{\NBIMO}{#1}{#200}}%

% north bimorphism with left name [1] and length 5000 units
\newcommand{\Nbimo}[1]{\Nbimov{#1}{50}}%

% north bimorphism with right name [1] and length [2]x100 units
\newcommand{\nbimOv}[2]{\vcasE{\NBIMO}{#1}{#200}}%

% north bimorphism with right name [1] and length 5000 units
\newcommand{\nbimO}[1]{\nbimOv{#1}{50}}%

% north isomorphism of length [1]x100 units
\newcommand{\nisov}[1]{\vcasE{\NAR}{\cong}{#100}}%

% north isomorphism of length 5000 units
\newcommand{\niso}{\nisov{50}}%

% north isomorphism with left name [1] and length [2]x100 units
\newcommand{\Nisov}[2]%
{\Vbicase{\NAR}{#1\hspace{-2pt}}{\hspace{-2pt}\cong}{#200}}%

% north isomorphism with left name [1] and length 5000 units
\newcommand{\Niso}[1]{\Nisov{#1}{50}}%

% pair of north arrows of length [1]x100 units
\newcommand{\nbiarv}[1]{\vbicase{\NBIAR}{#100}}%

% pair of north arrows of length 5000 units
\newcommand{\nbiar}{\nbiarv{50}}%

% pair of north arrows with names [1],[2] and length [3]x100 units
\newcommand{\Nbiarv}[3]{\Vbicase{\NBIAR}{#1}{#2}{#300}}%

% pair of north arrows with names [1],[2] and length 5000 units
\newcommand{\Nbiar}[2]{\Nbiarv{#1}{#2}{50}}%

% north equality of length [1]x100 units
\newcommand{\neqlv}[1]{\vbicase{\SEQL}{#100}}%

% north equality of length 50 pt
\newcommand{\neql}{\neqlv{50}}%

% north pair of adjoint arrows of length [1]x100 units
\newcommand{\nadjarv}[1]{\vbicase{\NADJAR}{#100}}%

% north pair of adjoint arrows of length 5000 units
\newcommand{\nadjar}{\nadjarv{50}}%

% north pair of adjoint arrows with names [1],[2] and length [3]x100 units
\newcommand{\Nadjarv}[3]{\Vbicase{\NADJAR}{#1}{#2}{#300}}%

% north pair of adjoint arrows with names [1],[2] and length 5000 units
\newcommand{\Nadjar}[2]{\Nadjarv{#1}{#2}{50}}%



% MACROS FOR  FIRST DIAGONAL PICTURES

% \fdcase{P}{f}{g} draws the picture P with names f, g
\newcommand{\fdcase}[3]{\begin{picture}(0,0)%
\put(0,-150){#1}%
\truex{200}\truey{600}\truez{600}%
\put(-\value{x},-\value{x}){\makebox(0,\value{z})[r]{${#2}$}}%
\put(\value{x},-\value{y}){\makebox(0,\value{z})[l]{${#3}$}}%
\end{picture}}%

% \fdbicase{P}{f}{g} draws the bipicture P with names f, g
\newcommand{\fdbicase}[3]{\begin{picture}(0,0)%
\put(0,-150){#1}%
\truex{800}\truey{50}%
\put(-\value{x},\value{y}){${#2}$}%
\truex{200}\truey{950}%
\put(\value{x},-\value{y}){${#3}$}%
\end{picture}}%





% NORTH-EAST ARROWS

% \NEAR draws a north-east arrow
\newcommand{\NEAR}{\begin{picture}(0,0)%
\put(-2900,-2900){\vector(1,1){5800}}%
\end{picture}}%

% \NEDOTAR draws a north-east dotted arrow
\newcommand{\NEDOTAR}%
{\truex{100}\truey{212}%
\NUMBEROFDOTS=5800%
\divide\NUMBEROFDOTS by \value{y}%
\begin{picture}(0,0)%
\multiput(-2900,-2900)(\value{y},\value{y}){\NUMBEROFDOTS}%
{\circle*{\value{x}}}%
\put(2900,2900){\vector(1,1){0}}%
\end{picture}}%

% \NEMONO draws a north-east monomorphism
\newcommand{\NEMONO}{\begin{picture}(0,0)%
\put(-2900,-2900){\vector(1,1){5800}}%
\put(-2900,-2900){\begin{picture}(0,0)%
\truex{141}%
\put(\value{x},\value{x}){\makebox(0,0){$\times$}}%
\end{picture}}\end{picture}}%

% \NEEPI draws a north-east epimorphism
\newcommand{\NEEPI}{\begin{picture}(0,0)%
\put(-2900,-2900){\vector(1,1){5800}}%
\put(2900,2900){\begin{picture}(0,0)%
\truex{545}%
\put(-\value{x},-\value{x}){\makebox(0,0){$\times$}}%
\end{picture}}\end{picture}}%

% \NEBIMO draws a north-east bimorphism
\newcommand{\NEBIMO}{\begin{picture}(0,0)%
\put(-2900,-2900){\vector(1,1){5800}}%
\put(2900,2900){\begin{picture}(0,0)%
\truex{545}%
\put(-\value{x},-\value{x}){\makebox(0,0){$\times$}}%
\end{picture}}
\put(-2900,-2900){\begin{picture}(0,0)%
\truex{141}%
\put(\value{x},\value{x}){\makebox(0,0){$\times$}}%
\end{picture}}\end{picture}}%

% \NEBIAR draws a pair of north-east arrows
\newcommand{\NEBIAR}{\begin{picture}(0,0)%
\put(-2900,-2900){\begin{picture}(0,0)%
\truex{141}%
\put(-\value{x},\value{x}){\vector(1,1){5800}}%
\put(\value{x},-\value{x}){\vector(1,1){5800}}%
\end{picture}}\end{picture}}%

% \NEEQL draws a north-east equality
\newcommand{\NEEQL}{\begin{picture}(0,0)%
\put(-2900,-2900){\begin{picture}(0,0)%
\truex{70}%
\put(-\value{x},\value{x}){\line(1,1){5800}}%
\put(\value{x},-\value{x}){\line(1,1){5800}}%
\end{picture}}\end{picture}}%

% \NEADJAR draws a north-east pair of adjoint arrows
\newcommand{\NEADJAR}{\begin{picture}(0,0)%
\put(-2900,-2900){\begin{picture}(0,0)%
\truex{141}%
\put(\value{x},-\value{x}){\vector(1,1){5800}}%
\end{picture}}%
\put(2900,2900){\begin{picture}(0,0)%
\truex{141}%
\put(-\value{x},\value{x}){\vector(-1,-1){5800}}%
\end{picture}}\end{picture}}%

% \NEARV{n} draws a north-east arrow of length nx100 units
\newcommand{\NEARV}[1]{\begin{picture}(0,0)%
\put(0,0){\makebox(0,0){\begin{picture}(#1,#1)%
\put(0,0){\vector(1,1){#1}}\end{picture}}}%
\end{picture}}%


% All the following commands draw north-east arrows of horizontal
% extent 5800 units and of formal dimensions (0,0).

% north-east arrow
\newcommand{\near}{\fdcase{\NEAR}{}{}}%

% north-east arrow with upper name [1]
\newcommand{\Near}[1]{\fdcase{\NEAR}{#1}{}}%

% north-east arrow with lower name [1]
\newcommand{\neaR}[1]{\fdcase{\NEAR}{}{#1}}%

% north-east dotted arrow
\newcommand{\nedotar}{\fdcase{\NEDOTAR}{}{}}%

% north-east dotted arrow with upper name [1]
\newcommand{\Nedotar}[1]{\fdcase{\NEDOTAR}{#1}{}}%

% north-east dotted arrow with lower name [1]
\newcommand{\nedotaR}[1]{\fdcase{\NEDOTAR}{}{#1}}%

% north-east monomorphism
\newcommand{\nemono}{\fdcase{\NEMONO}{}{}}%

% north-est monomorphism with upper name [1]
\newcommand{\Nemono}[1]{\fdcase{\NEMONO}{#1}{}}%

% north-east monomorphism with lower name [1]
\newcommand{\nemonO}[1]{\fdcase{\NEMONO}{}{#1}}%

% north-east epimorphism
\newcommand{\neepi}{\fdcase{\NEEPI}{}{}}%

% north-east epimorphism with upper name [1]
\newcommand{\Neepi}[1]{\fdcase{\NEEPI}{#1}{}}%

% north-east epimorphism with lower name [1]
\newcommand{\neepI}[1]{\fdcase{\NEEPI}{}{#1}}%

% north-east bimorphism
\newcommand{\nebimo}{\fdcase{\NEBIMO}{}{}}%

% north-east bimorphism with upper name [1]
\newcommand{\Nebimo}[1]{\fdcase{\NEBIMO}{#1}{}}%

% north-east bimorphism with lower name [1]
\newcommand{\nebimO}[1]{\fdcase{\NEBIMO}{}{#1}}%

% north-east isomorphism
\newcommand{\neiso}{\fdcase{\NEAR}{\hspace{-2pt}\cong}{}}%

% north-east isomorphism with name [1]
\newcommand{\Neiso}[1]{\fdcase{\NEAR}{\hspace{-2pt}\cong}{#1}}%

% pair of north-east arrows
\newcommand{\nebiar}{\fdbicase{\NEBIAR}{}{}}%

% pair of north-east arrows with names [1], [2]
\newcommand{\Nebiar}[2]{\fdbicase{\NEBIAR}{#1}{#2}}%

% north-east equality
\newcommand{\neeql}{\fdbicase{\NEEQL}{}{}}%

% north-east pair of adjoint arrows
\newcommand{\neadjar}{\fdbicase{\NEADJAR}{}{}}%

% north-east pair of adjoint arrows with names [1], [2]
\newcommand{\Neadjar}[2]{\fdbicase{\NEADJAR}{#1}{#2}}%


% the following commands produce a north-east arrow of variable length
% and of formal dimensions (0,0).

% north-east arrow of horizontal length [1]x100 units
\newcommand{\nearv}[1]{\fdcase{\NEARV{#100}}{}{}}%

% north-east arrow with upper name [1] and horizontal length [2]x100 units
\newcommand{\Nearv}[2]{\fdcase{\NEARV{#200}}{#1}{}}%

% north-east arrow with lower name [1] and horizontal length [2]x100 units
\newcommand{\neaRv}[2]{\fdcase{\NEARV{#200}}{}{#1}}%





% SOUTH-WEST ARROWS

% \SWAR draws a south-west arrow
\newcommand{\SWAR}{\begin{picture}(0,0)%
\put(2900,2900){\vector(-1,-1){5800}}%
\end{picture}}%

% \SWDOTAR draws a south-west dotted arrow
\newcommand{\SWDOTAR}%
{\truex{100}\truey{212}%
\NUMBEROFDOTS=5800%
\divide\NUMBEROFDOTS by \value{y}%
\begin{picture}(0,0)%
\multiput(2900,2900)(-\value{y},-\value{y}){\NUMBEROFDOTS}%
{\circle*{\value{x}}}%
\put(-2900,-2900){\vector(-1,-1){0}}%
\end{picture}}%

% \SWMONO draws a south-west monomorphism
\newcommand{\SWMONO}{\begin{picture}(0,0)%
\put(2900,2900){\vector(-1,-1){5800}}%
\put(2900,2900){\begin{picture}(0,0)%
\truex{141}%
\put(-\value{x},-\value{x}){\makebox(0,0){$\times$}}%
\end{picture}}\end{picture}}%

% \SWEPI draws a south-west epimorphism
\newcommand{\SWEPI}{\begin{picture}(0,0)%
\put(2900,2900){\vector(-1,-1){5800}}%
\put(-2900,-2900){\begin{picture}(0,0)%
\truex{525}%
\put(\value{x},\value{x}){\makebox(0,0){$\times$}}%
\end{picture}}\end{picture}}%

% \SWBIMO draws a south-west bimorphism
\newcommand{\SWBIMO}{\begin{picture}(0,0)%
\put(2900,2900){\vector(-1,-1){5800}}%
\put(2900,2900){\begin{picture}(0,0)%
\truex{141}%
\put(-\value{x},-\value{x}){\makebox(0,0){$\times$}}%
\end{picture}}%
\put(-2900,-2900){\begin{picture}(0,0)%
\truex{525}%
\put(\value{x},\value{x}){\makebox(0,0){$\times$}}%
\end{picture}}\end{picture}}%

% \SWBIAR draws a pair of south-west arrows
\newcommand{\SWBIAR}{\begin{picture}(0,0)%
\put(2900,2900){\begin{picture}(0,0)%
\truex{141}%
\put(\value{x},-\value{x}){\vector(-1,-1){5800}}%
\put(-\value{x},\value{x}){\vector(-1,-1){5800}}%
\end{picture}}\end{picture}}%

% \SWADJAR draws a south-west pair of adjoint arrows
\newcommand{\SWADJAR}{\begin{picture}(0,0)%
\put(-2900,-2900){\begin{picture}(0,0)%
\truex{141}%
\put(-\value{x},\value{x}){\vector(1,1){5800}}%
\end{picture}}%
\put(2900,2900){\begin{picture}(0,0)%
\truex{141}%
\put(\value{x},-\value{x}){\vector(-1,-1){5800}}%
\end{picture}}\end{picture}}%

% \SWARV{n} draws a south-west arrow of length nx100 units
\newcommand{\SWARV}[1]{\begin{picture}(0,0)%
\put(0,0){\makebox(0,0){\begin{picture}(#1,#1)%
\put(#1,#1){\vector(-1,-1){#1}}\end{picture}}}%
\end{picture}}%


% All the following commands draw south-west arrows of horizontal
% extent 5800 units and formal dimensions (0,0).

% south-west arrow
\newcommand{\swar}{\fdcase{\SWAR}{}{}}%

% south-west arrow with upper name [1]
\newcommand{\Swar}[1]{\fdcase{\SWAR}{#1}{}}%

% south-west arrow with lower name [1]
\newcommand{\swaR}[1]{\fdcase{\SWAR}{}{#1}}%

% south-west dotted arrow
\newcommand{\swdotar}{\fdcase{\SWDOTAR}{}{}}%

% south-west dotted arrow with upper name [1]
\newcommand{\Swdotar}[1]{\fdcase{\SWDOTAR}{#1}{}}%

% south-west dotted arrow with lower name [1]
\newcommand{\swdotaR}[1]{\fdcase{\SWDOTAR}{}{#1}}%

% south-west monomorphism
\newcommand{\swmono}{\fdcase{\SWMONO}{}{}}%

% north-est monomorphism with upper name [1]
\newcommand{\Swmono}[1]{\fdcase{\SWMONO}{#1}{}}%

% south-west monomorphism with lower name [1]
\newcommand{\swmonO}[1]{\fdcase{\SWMONO}{}{#1}}%

% south-west epimorphism
\newcommand{\swepi}{\fdcase{\SWEPI}{}{}}%

% south-west epimorphism with upper name [1]
\newcommand{\Swepi}[1]{\fdcase{\SWEPI}{#1}{}}%

% south-west epimorphism with lower name [1]
\newcommand{\swepI}[1]{\fdcase{\SWEPI}{}{#1}}%

% south-west bimorphism
\newcommand{\swbimo}{\fdcase{\SWBIMO}{}{}}%

% south-west bimorphism with upper name [1]
\newcommand{\Swbimo}[1]{\fdcase{\SWBIMO}{#1}{}}%

% south-west bimorphism with lower name [1]
\newcommand{\swbimO}[1]{\fdcase{\SWBIMO}{}{#1}}%

% south-west isomorphism
\newcommand{\swiso}{\fdcase{\SWAR}{\hspace{-2pt}\cong}{}}%

% south-west isomorphism with name [1]
\newcommand{\Swiso}[1]{\fdcase{\SWAR}{\hspace{-2pt}\cong}{#1}}%

% pair of south-west arrows
\newcommand{\swbiar}{\fdbicase{\SWBIAR}{}{}}%

% pair of south-west arrows with names [1], [2]
\newcommand{\Swbiar}[2]{\fdbicase{\SWBIAR}{#1}{#2}}%

% south-west equality
\newcommand{\sweql}{\fdbicase{\NEEQL}{}{}}%

% south-west pair of adjoint arrows
\newcommand{\swadjar}{\fdbicase{\SWADJAR}{}{}}%

% south-west pair of adjoint arrows with names [1], [2]
\newcommand{\Swadjar}[2]{\fdbicase{\SWADJAR}{#1}{#2}}%


% the following commands produce a south-west arrow of variable length
% and formal dimensions (0,0).

% south-west arrow of horizontal length [1]x100 units
\newcommand{\swarv}[1]{\fdcase{\SWARV{#100}}{}{}}%

% south-west arrow with upper name [1] and horizontal length [2]x100 units
\newcommand{\Swarv}[2]{\fdcase{\SWARV{#200}}{#1}{}}%

% south-west arrow with lower name [1] and horizontal length [2]x100 units
\newcommand{\swaRv}[2]{\fdcase{\SWARV{#200}}{}{#1}}%





% MACROS FOR  SECOND DIAGONAL PICTURES

% \sdcase{P}{f}{g} draws the picture P with names f, g
\newcommand{\sdcase}[3]{\begin{picture}(0,0)%
\put(0,-150){#1}%
\truex{100}\truez{600}%
\put(\value{x},\value{x}){\makebox(0,\value{z})[l]{${#2}$}}%
\truex{300}\truey{800}%
\put(-\value{x},-\value{y}){\makebox(0,\value{z})[r]{${#3}$}}%
\end{picture}}%

% \sdbicase{P}{f}{g} draws the bipicture P with names f, g
\newcommand{\sdbicase}[3]{\begin{picture}(0,0)%
\put(0,-150){#1}%
\truex{250}\truey{600}\truez{850}%
\put(\value{x},\value{x}){\makebox(0,\value{y})[l]{${#2}$}}%
\put(-\value{x},-\value{z}){\makebox(0,\value{y})[r]{${#3}$}}%
\end{picture}}%





% SOUT-EAST ARROWS

% \SEAR draws a south-east arrow
\newcommand{\SEAR}{\begin{picture}(0,0)%
\put(-2900,2900){\vector(1,-1){5800}}%
\end{picture}}%

% \SEDOTAR draws a south-east dotted arrow
\newcommand{\SEDOTAR}%
{\truex{100}\truey{212}%
\NUMBEROFDOTS=5800%
\divide\NUMBEROFDOTS by \value{y}%
\begin{picture}(0,0)%
\multiput(-2900,2900)(\value{y},-\value{y}){\NUMBEROFDOTS}%
{\circle*{\value{x}}}%
\put(2900,-2900){\vector(1,-1){0}}%
\end{picture}}%

% \SEMONO draws a south-east monomorphism
\newcommand{\SEMONO}{\begin{picture}(0,0)%
\put(-2900,2900){\vector(1,-1){5800}}%
\put(-2900,2900){\begin{picture}(0,0)%
\truex{141}%
\put(\value{x},-\value{x}){\makebox(0,0){$\times$}}%
\end{picture}}\end{picture}}%

% \SEEPI draws a south-east epimorphism
\newcommand{\SEEPI}{\begin{picture}(0,0)%
\put(-2900,2900){\vector(1,-1){5800}}%
\put(2900,-2900){\begin{picture}(0,0)%
\truex{525}%
\put(-\value{x},\value{x}){\makebox(0,0){$\times$}}%
\end{picture}}\end{picture}}%

% \SEBIMO draws a south-east bimorphism
\newcommand{\SEBIMO}{\begin{picture}(0,0)%
\put(-2900,2900){\vector(1,-1){5800}}%
\put(-2900,2900){\begin{picture}(0,0)%
\truex{141}%
\put(\value{x},-\value{x}){\makebox(0,0){$\times$}}%
\end{picture}}%
\put(2900,-2900){\begin{picture}(0,0)%
\truex{525}%
\put(-\value{x},\value{x}){\makebox(0,0){$\times$}}%
\end{picture}}\end{picture}}%

% \SEBIAR draws a pair of south-east arrows
\newcommand{\SEBIAR}{\begin{picture}(0,0)%
\put(-2900,2900){\begin{picture}(0,0)%
\truex{141}
\put(-\value{x},-\value{x}){\vector(1,-1){5800}}%
\put(\value{x},\value{x}){\vector(1,-1){5800}}%
\end{picture}}\end{picture}}%

% \SEEQL draws a south-east equality
\newcommand{\SEEQL}{\begin{picture}(0,0)%
\put(-2900,2900){\begin{picture}(0,0)%
\truex{70}%
\put(-\value{x},-\value{x}){\line(1,-1){5800}}%
\put(\value{x},\value{x}){\line(1,-1){5800}}%
\end{picture}}\end{picture}}%


% \SEADJAR draws a south-east pair of adjoint arrows
\newcommand{\SEADJAR}{\begin{picture}(0,0)%
\put(-2900,2900){\begin{picture}(0,0)%
\truex{141}%
\put(-\value{x},-\value{x}){\vector(1,-1){5800}}%
\end{picture}}%
\put(2900,-2900){\begin{picture}(0,0)%
\truex{141}%
\put(\value{x},\value{x}){\vector(-1,1){5800}}%
\end{picture}}\end{picture}}%

% \SEARV{n} draws a south-east arrow of length nx100 units
\newcommand{\SEARV}[1]{\begin{picture}(0,0)%
\put(0,0){\makebox(0,0){\begin{picture}(#1,#1)%
\put(0,#1){\vector(1,-1){#1}}\end{picture}}}%
\end{picture}}%


% All the following commands draw south-east arrows of horizontal
% extent 5800 units and of formal dimensions (0,0).

% south-east arrow
\newcommand{\sear}{\sdcase{\SEAR}{}{}}%

% south-east arrow with upper name [1]
\newcommand{\Sear}[1]{\sdcase{\SEAR}{#1}{}}%

% south-east arrow with lower name [1]
\newcommand{\seaR}[1]{\sdcase{\SEAR}{}{#1}}%

% south-east dotted arrow
\newcommand{\sedotar}{\sdcase{\SEDOTAR}{}{}}%

% south-east dotted arrow with upper name [1]
\newcommand{\Sedotar}[1]{\sdcase{\SEDOTAR}{#1}{}}%

% south-east dotted arrow with lower name [1]
\newcommand{\sedotaR}[1]{\sdcase{\SEDOTAR}{}{#1}}%

% south-east monomorphism
\newcommand{\semono}{\sdcase{\SEMONO}{}{}}%

% south-east monomorphism with upper name [1]
\newcommand{\Semono}[1]{\sdcase{\SEMONO}{#1}{}}%

% south-east monomorphism with lower name [1]
\newcommand{\semonO}[1]{\sdcase{\SEMONO}{}{#1}}%

% south-east epimorphism
\newcommand{\seepi}{\sdcase{\SEEPI}{}{}}%

% south-east epimorphism with upper name [1]
\newcommand{\Seepi}[1]{\sdcase{\SEEPI}{#1}{}}%

% south-east epimorphism with lower name [1]
\newcommand{\seepI}[1]{\sdcase{\SEEPI}{}{#1}}%

% south-east bimorphism
\newcommand{\sebimo}{\sdcase{\SEBIMO}{}{}}%

% south-east bimorphism with upper name [1]
\newcommand{\Sebimo}[1]{\sdcase{\SEBIMO}{#1}{}}%

% south-east bimorphism with lower name [1]
\newcommand{\sebimO}[1]{\sdcase{\SEBIMO}{}{#1}}%

% south-east isomorphism
\newcommand{\seiso}{\sdcase{\SEAR}{\hspace{-2pt}\cong}{}}%

% south-east isomorphism with name [1]
\newcommand{\Seiso}[1]{\sdcase{\SEAR}{\hspace{-2pt}\cong}{#1}}%

% pair of south-east arrows
\newcommand{\sebiar}{\sdbicase{\SEBIAR}{}{}}%

% pair of south-east arrows with names [1], [2]
\newcommand{\Sebiar}[2]{\sdbicase{\SEBIAR}{#1}{#2}}%

% south-east equality
\newcommand{\seeql}{\sdbicase{\SEEQL}{}{}}%

% south-east pair of adjoint arrows
\newcommand{\seadjar}{\sdbicase{\SEADJAR}{}{}}%

% south-east pair of adjoint arrows with names [1], [2]
\newcommand{\Seadjar}[2]{\sdbicase{\SEADJAR}{#1}{#2}}%


% the following commands produce a south-east arrow of variable length
% and of formal dimensions (0,0).


% south-east arrow of horizontal length [1]x100 units
\newcommand{\searv}[1]{\sdcase{\SEARV{#100}}{}{}}%

% south-east arrow with upper name [1] and horizontal length [2]x100 units
\newcommand{\Searv}[2]{\sdcase{\SEARV{#200}}{#1}{}}%

% south-east arrow with lower name [1] and horizontal length [2]x100 units
\newcommand{\seaRv}[2]{\sdcase{\SEARV{#200}}{}{#1}}%




% NORTH-WEST ARROWS

% \NWAR draws a north-west arrow
\newcommand{\NWAR}{\begin{picture}(0,0)%
\put(2900,-2900){\vector(-1,1){5800}}%
\end{picture}}%

% \NWDOTAR draws a north-west dotted arrow
\newcommand{\NWDOTAR}%
{\truex{100}\truey{212}%
\NUMBEROFDOTS=5800%
\divide\NUMBEROFDOTS by \value{y}%
\begin{picture}(0,0)%
\multiput(2900,-2900)(-\value{y},\value{y}){\NUMBEROFDOTS}%
{\circle*{\value{x}}}%
\put(-2900,2900){\vector(-1,1){0}}%
\end{picture}}%

% \NWMONO draws a north-west monomorphism
\newcommand{\NWMONO}{\begin{picture}(0,0)%
\put(2900,-2900){\vector(-1,1){5800}}%
\put(2900,-2900){\begin{picture}(0,0)%
\truex{141}%
\put(-\value{x},\value{x}){\makebox(0,0){$\times$}}%
\end{picture}}\end{picture}}%

% \NWEPI draws a north-west epimorphism
\newcommand{\NWEPI}{\begin{picture}(0,0)%
\put(2900,-2900){\vector(-1,1){5800}}%
\put(-2900,2900){\begin{picture}(0,0)%
\truex{525}%
\put(\value{x},-\value{x}){\makebox(0,0){$\times$}}%
\end{picture}}\end{picture}}%

% \NWBIMO draws a north-west bimorphism
\newcommand{\NWBIMO}{\begin{picture}(0,0)%
\put(2900,-2900){\vector(-1,1){5800}}%
\put(2900,-2900){\begin{picture}(0,0)%
\truex{141}%
\put(-\value{x},\value{x}){\makebox(0,0){$\times$}}%
\end{picture}}%
\put(-2900,2900){\begin{picture}(0,0)%
\truex{525}%
\put(\value{x},-\value{x}){\makebox(0,0){$\times$}}%
\end{picture}}\end{picture}}%

% \NWBIAR draws a pair of north-west arrows
\newcommand{\NWBIAR}{\begin{picture}(0,0)%
\put(2900,-2900){\begin{picture}(0,0)%
\truex{141}%
\put(\value{x},\value{x}){\vector(-1,1){5800}}%
\end{picture}}%
\put(2900,-2900){\begin{picture}(0,0)%
\truex{141}
\put(-\value{x},-\value{x}){\vector(-1,1){5800}}%
\end{picture}}\end{picture}}%

% \NWADJAR draws a north-west pair of adjoint arrows
\newcommand{\NWADJAR}{\begin{picture}(0,0)%
\put(-2900,2900){\begin{picture}(0,0)%
\truex{141}%
\put(\value{x},\value{x}){\vector(1,-1){5800}}%
\end{picture}}%
\put(2900,-2900){\begin{picture}(0,0)%
\truex{141}%
\put(-\value{x},-\value{x}){\vector(-1,1){5800}}%
\end{picture}}\end{picture}}%

% \NWARV{n} draws a north-west arrow of length nx100 units
\newcommand{\NWARV}[1]{\begin{picture}(0,0)%
\put(0,0){\makebox(0,0){\begin{picture}(#1,#1)%
\put(#1,0){\vector(-1,1){#1}}\end{picture}}}%
\end{picture}}%


% All the following commands draw north-west arrows of horizontal
% extent 5800 units and of formal dimensions (0,0).

% north-west arrow
\newcommand{\nwar}{\sdcase{\NWAR}{}{}}%

% north-west arrow with upper name [1]
\newcommand{\Nwar}[1]{\sdcase{\NWAR}{#1}{}}%

% north-west arrow with lower name [1]
\newcommand{\nwaR}[1]{\sdcase{\NWAR}{}{#1}}%

% north-west dotted arrow
\newcommand{\nwdotar}{\sdcase{\NWDOTAR}{}{}}%

% north-west dotted arrow with upper name [1]
\newcommand{\Nwdotar}[1]{\sdcase{\NWDOTAR}{#1}{}}%

% north-west dotted arrow with lower name [1]
\newcommand{\nwdotaR}[1]{\sdcase{\NWDOTAR}{}{#1}}%

% north-west monomorphism
\newcommand{\nwmono}{\sdcase{\NWMONO}{}{}}%

% north-west monomorphism with upper name [1]
\newcommand{\Nwmono}[1]{\sdcase{\NWMONO}{#1}{}}%

% north-west monomorphism with lower name [1]
\newcommand{\nwmonO}[1]{\sdcase{\NWMONO}{}{#1}}%

% north-west epimorphism
\newcommand{\nwepi}{\sdcase{\NWEPI}{}{}}%

% north-west epimorphism with upper name [1]
\newcommand{\Nwepi}[1]{\sdcase{\NWEPI}{#1}{}}%

% north-west epimorphism with lower name [1]
\newcommand{\nwepI}[1]{\sdcase{\NWEPI}{}{#1}}%

% north-west bimorphism
\newcommand{\nwbimo}{\sdcase{\NWBIMO}{}{}}%

% north-west bimorphism with upper name [1]
\newcommand{\Nwbimo}[1]{\sdcase{\NWBIMO}{#1}{}}%

% north-west bimorphism with lower name [1]
\newcommand{\nwbimO}[1]{\sdcase{\NWBIMO}{}{#1}}%

% north-west isomorphism
\newcommand{\nwiso}{\sdcase{\NWAR}{\hspace{-2pt}\cong}{}}%

% north-west isomorphism with name [1]
\newcommand{\Nwiso}[1]{\sdcase{\NWAR}{\hspace{-2pt}\cong}{#1}}%

% pair of north-west arrows
\newcommand{\nwbiar}{\sdbicase{\NWBIAR}{}{}}%

% pair of north-west arrows with names [1], [2]
\newcommand{\Nwbiar}[2]{\sdbicase{\NWBIAR}{#1}{#2}}%

% north-west equality
\newcommand{\nweql}{\sdbicase{\SEEQL}{}{}}%

% north-west pair of adjoint arrows
\newcommand{\nwadjar}{\sdbicase{\NWADJAR}{}{}}%

% north-west pair of adjoint arrows with names [1], [2]
\newcommand{\Nwadjar}[2]{\sdbicase{\NWADJAR}{#1}{#2}}%


% the following commands produce a north-west arrow of variable length
% and of formal dimensions (0,0).

% north-west arrow of horizontal length [1]x100 units
\newcommand{\nwarv}[1]{\sdcase{\NWARV{#100}}{}{}}%

% north-west arrow with upper name [1] and horizontal length [2]x100 units
\newcommand{\Nwarv}[2]{\sdcase{\NWARV{#200}}{#1}{}}%

% north-west arrow with lower name [1] and horizontal length [2]x100 units
\newcommand{\nwaRv}[2]{\sdcase{\NWARV{#200}}{}{#1}}%




% HORIZONTAL RECTANGLE DIAGONAL ARROWS

% All the following commands produce arrows of horizontal extent 13200
% units, oriented in the directions of the diagonals of an horizontal
% rectangle of sides 2, 1. The results are positioned in a box of width
% 0pt and height 0pt.

% \ENEAR{f}{g} draws a east-north-east arrow with names f, g
\newcommand{\ENEAR}[2]%
{\makebox[0pt]{\begin{picture}(0,0)%
\put(0,-150){\makebox(0,0){\begin{picture}(0,0)%
\put(-6600,-3300){\vector(2,1){13200}}%
\truex{200}\truey{800}\truez{600}%
\put(-\value{x},\value{x}){\makebox(0,\value{z})[r]{${#1}$}}%
\put(\value{x},-\value{y}){\makebox(0,\value{z})[l]{${#2}$}}%
\end{picture}}}\end{picture}}}%

% east-north-east arrow
\newcommand{\enear}{\ENEAR{}{}}%

% east-north-east arrow with upper name [1]
\newcommand{\Enear}[1]{\ENEAR{#1}{}}%

% east-north-east arrow with lower name [1]
\newcommand{\eneaR}[1]{\ENEAR{}{#1}}%

% \ESEAR{f}{g} draws an east-south-east arrow with names f, g
\newcommand{\ESEAR}[2]%
{\makebox[0pt]{\begin{picture}(0,0)%
\put(0,-150){\makebox(0,0){\begin{picture}(0,0)%
\put(-6600,3300){\vector(2,-1){13200}}%
\truex{200}\truey{800}\truez{600}%
\put(\value{x},\value{x}){\makebox(0,\value{z})[l]{${#1}$}}%
\put(-\value{x},-\value{y}){\makebox(0,\value{z})[r]{${#2}$}}%
\end{picture}}}\end{picture}}}%

% east-south-east arrow
\newcommand{\esear}{\ESEAR{}{}}%

% east-south-east arrow with upper name [1]
\newcommand{\Esear}[1]{\ESEAR{#1}{}}%

% east-south-east arrow with lower name [1]
\newcommand{\eseaR}[1]{\ESEAR{}{#1}}%

% \WNWAR{f}{g} draws a west-north-west arrow with names f, g
\newcommand{\WNWAR}[2]%
{\makebox[0pt]{\begin{picture}(0,0)%
\put(0,-150){\makebox(0,0){\begin{picture}(0,0)%
\put(6600,-3300){\vector(-2,1){13200}}%
\truex{200}\truey{800}\truez{600}%
\put(\value{x},\value{x}){\makebox(0,\value{z})[l]{${#1}$}}%
\put(-\value{x},-\value{y}){\makebox(0,\value{z})[r]{${#2}$}}%
\end{picture}}}\end{picture}}}%

% west-north-west arrow
\newcommand{\wnwar}{\WNWAR{}{}}%

% west-north-west arrow with upper name [1]
\newcommand{\Wnwar}[1]{\WNWAR{#1}{}}%

% west-north-west arrow with lower name [1]
\newcommand{\wnwaR}[1]{\WNWAR{}{#1}}%

% \WSWAR{f}{g} draws a west-south-west arrow with names f, g
\newcommand{\WSWAR}[2]%
{\makebox[0pt]{\begin{picture}(0,0)%
\put(0,-150){\makebox(0,0){\begin{picture}(0,0)%
\put(6600,3300){\vector(-2,-1){13200}}%
\truex{200}\truey{800}\truez{600}%
\put(-\value{x},\value{x}){\makebox(0,\value{z})[r]{${#1}$}}%
\put(\value{x},-\value{y}){\makebox(0,\value{z})[l]{${#2}$}}%
\end{picture}}}\end{picture}}}%

% west-south-west arrow
\newcommand{\wswar}{\WSWAR{}{}}%

% west-south-west arrow with upper name [1]
\newcommand{\Wswar}[1]{\WSWAR{#1}{}}%

% west-south-west arrow with lower name [1]
\newcommand{\wswaR}[1]{\WSWAR{}{#1}}%




% VERTICAL RECTANGLE DIAGONAL ARROWS

% All the following commands produce arrows of horizontal extent 6600
% units, oriented in the directions of the diagonals of a vertical
% rectangle of sides 1, 2. The results are positioned in a box of width
% 0pt and height 0pt.


% \NNEAR{f}{g} draws a north-north-east arrow with names f, g
\newcommand{\NNEAR}[2]%
{\raisebox{-1pt}[0pt][0pt]{\begin{picture}(0,0)%
\put(0,0){\makebox(0,0){\begin{picture}(0,0)%
\put(-3300,-6600){\vector(1,2){6600}}%
\truex{100}\truez{600}%
\put(-\value{x},\value{x}){\makebox(0,\value{z})[r]{${#1}$}}%
\put(\value{x},-\value{z}){\makebox(0,\value{z})[l]{${#2}$}}%
\end{picture}}}\end{picture}}}%

% north-north-east arrow
\newcommand{\nnear}{\NNEAR{}{}}%

% north-north-east arrow with upper name [1]
\newcommand{\Nnear}[1]{\NNEAR{#1}{}}%

% north-north-east arrow with lower name [1]
\newcommand{\nneaR}[1]{\NNEAR{}{#1}}%

% \SSWAR{f}{g} draws a south-south-west arrow with names f, g
\newcommand{\SSWAR}[2]%
{\raisebox{-1pt}[0pt][0pt]{\begin{picture}(0,0)%
\put(0,0){\makebox(0,0){\begin{picture}(0,0)%
\put(3300,6600){\vector(-1,-2){6600}}%
\truex{100}\truez{600}%
\put(-\value{x},\value{x}){\makebox(0,\value{z})[r]{${#1}$}}%
\put(\value{x},-\value{z}){\makebox(0,\value{z})[l]{${#2}$}}%
\end{picture}}}\end{picture}}}%

% south-south-west arrow
\newcommand{\sswar}{\SSWAR{}{}}%

% south-south-west arrow with upper name [1]
\newcommand{\Sswar}[1]{\SSWAR{#1}{}}%

% south-south-west arrow with lower name [1]
\newcommand{\sswaR}[1]{\SSWAR{}{#1}}%

% \SSEAR{f}{g} draws a south-south-east arrow with names f, g
\newcommand{\SSEAR}[2]%
{\raisebox{-1pt}[0pt][0pt]{\begin{picture}(0,0)%
\put(0,0){\makebox(0,0){\begin{picture}(0,0)%
\put(-3300,6600){\vector(1,-2){6600}}%
\truex{200}\truez{600}%
\put(\value{x},\value{x}){\makebox(0,\value{z})[l]{${#1}$}}%
\put(-\value{x},-\value{z}){\makebox(0,\value{z})[r]{${#2}$}}%
\end{picture}}}\end{picture}}}%

% south-south-east arrow
\newcommand{\ssear}{\SSEAR{}{}}%

% south-south-east arrow with upper name [1]
\newcommand{\Ssear}[1]{\SSEAR{#1}{}}%

% south-south-east arrow with lower name [1]
\newcommand{\sseaR}[1]{\SSEAR{}{#1}}%

% \NNWAR{f}{g} draws a north-north-west arrow with names f, g
\newcommand{\NNWAR}[2]%
{\raisebox{-1pt}[0pt][0pt]{\begin{picture}(0,0)%
\put(0,0){\makebox(0,0){\begin{picture}(0,0)%
\put(3300,-6600){\vector(-1,2){6600}}%
\truex{200}\truez{600}%
\put(\value{x},\value{x}){\makebox(0,\value{z})[l]{${#1}$}}%
\put(-\value{x},-\value{z}){\makebox(0,\value{z})[r]{${#2}$}}%
\end{picture}}}\end{picture}}}%

% north-north-west arrow
\newcommand{\nnwar}{\NNWAR{}{}}%

% north-north-west arrow with upper name [1]
\newcommand{\Nnwar}[1]{\NNWAR{#1}{}}%

% north-north-west arrow with lower name [1]
\newcommand{\nnwaR}[1]{\NNWAR{}{#1}}%





% HORIZONTAL CURVED ARROWS

% The following commands produce horizontal curved arrows
% positioned in a box of width 0pt and height 0pt.

% North\East curved arrow with name [1] and length [2]x100 units
\newcommand{\Necurve}[2]%
{\begin{picture}(0,0)%
\truex{1300}\truey{2000}\truez{200}%
\put(0,\value{x}){\oval(#200,\value{y})[t]}%
\put(0,\value{x}){\makebox(0,0){\begin{picture}(#200,0)%
\put(#200,0){\vector(0,-1){\value{z}}}%
\put(0,0){\line(0,-1){\value{z}}}\end{picture}}}%
\truex{2500}%
\put(0,\value{x}){\makebox(0,0)[b]{${#1}$}}%
\end{picture}}%

% North\East curved arrow of length [1]x100 units
\newcommand{\necurve}[1]{\Necurve{}{#1}}%

% North\West curved arrow with name [1] and length [2]x100 units
\newcommand{\Nwcurve}[2]%
{\begin{picture}(0,0)%
\truex{1300}\truey{2000}\truez{200}%
\put(0,\value{x}){\oval(#200,\value{y})[t]}%
\put(0,\value{x}){\makebox(0,0){\begin{picture}(#200,0)%
\put(#200,0){\line(0,-1){\value{z}}}%
\put(0,0){\vector(0,-1){\value{z}}}\end{picture}}}%
\truex{2500}%
\put(0,\value{x}){\makebox(0,0)[b]{${#1}$}}%
\end{picture}}%

% North\West curved arrow of length [1]x100 units
\newcommand{\nwcurve}[1]{\Nwcurve{}{#1}}%

% South\East curved arrow with name [1] and length [2]x100 units
\newcommand{\Securve}[2]%
{\begin{picture}(0,0)%
\truex{1300}\truey{2000}\truez{200}%
\put(0,-\value{x}){\oval(#200,\value{y})[b]}%
\put(0,-\value{x}){\makebox(0,0){\begin{picture}(#200,0)%
\put(#200,0){\vector(0,1){\value{z}}}%
\put(0,0){\line(0,1){\value{z}}}\end{picture}}}%
\truex{2500}%
\put(0,-\value{x}){\makebox(0,0)[t]{${#1}$}}%
\end{picture}}%

% South\East curved arrow of length [1]x100 units
\newcommand{\securve}[1]{\Securve{}{#1}}%

% South\West curved arrow with name [1] and length [2]x100 units
\newcommand{\Swcurve}[2]%
{\begin{picture}(0,0)%
\truex{1300}\truey{2000}\truez{200}%
\put(0,-\value{x}){\oval(#200,\value{y})[b]}%
\put(0,-\value{x}){\makebox(0,0){\begin{picture}(#200,0)%
\put(#200,0){\line(0,1){\value{z}}}%
\put(0,0){\vector(0,1){\value{z}}}\end{picture}}}%
\truex{2500}%
\put(0,-\value{x}){\makebox(0,0)[t]{${#1}$}}%
\end{picture}}%

% South\West curved arrow of length [1]x100 units
\newcommand{\swcurve}[1]{\Swcurve{}{#1}}%






% VERTICAL CURVED ARROWS

% The following commands produce vertical curved arrows
% positioned in a box of width 0pt and height 0pt

% East\South curved arrow with name [1] and length [2]x100 units
\newcommand{\Escurve}[2]%
{\begin{picture}(0,0)%
\truex{1400}\truey{2000}\truez{200}%
\put(\value{x},0){\oval(\value{y},#200)[r]}%
\put(\value{x},0){\makebox(0,0){\begin{picture}(0,#200)%
\put(0,0){\vector(-1,0){\value{z}}}%
\put(0,#200){\line(-1,0){\value{z}}}\end{picture}}}%
\truex{2500}%
\put(\value{x},0){\makebox(0,0)[l]{${#1}$}}%
\end{picture}}%

% East\South curved arrow of length [1]x100 units
\newcommand{\escurve}[1]{\Escurve{}{#1}}%

% East\North curved arrow with name [1] and length [2]x100 units
\newcommand{\Encurve}[2]%
{\begin{picture}(0,0)%
\truex{1400}\truey{2000}\truez{200}%
\put(\value{x},0){\oval(\value{y},#200)[r]}%
\put(\value{x},0){\makebox(0,0){\begin{picture}(0,#200)%
\put(0,0){\line(-1,0){\value{z}}}%
\put(0,#200){\vector(-1,0){\value{z}}}\end{picture}}}%
\truex{2500}%
\put(\value{x},0){\makebox(0,0)[l]{${#1}$}}%
\end{picture}}%

% East\North curved arrow of length [1]x100 units
\newcommand{\encurve}[1]{\Encurve{}{#1}}%

% West\South curved arrow with name [1] and length [2]x100 units
\newcommand{\Wscurve}[2]%
{\begin{picture}(0,0)%
\truex{1300}\truey{2000}\truez{200}%
\put(-\value{x},0){\oval(\value{y},#200)[l]}%
\put(-\value{x},0){\makebox(0,0){\begin{picture}(0,#200)%
\put(0,0){\vector(1,0){\value{z}}}%
\put(0,#200){\line(1,0){\value{z}}}\end{picture}}}%
\truex{2400}%
\put(-\value{x},0){\makebox(0,0)[r]{${#1}$}}%
\end{picture}}%

% West\South curved arrow of length [1]x100 units
\newcommand{\wscurve}[1]{\Wscurve{}{#1}}%

% West\North curved arrow with name [1] and length [2]x100 units
\newcommand{\Wncurve}[2]%
{\begin{picture}(0,0)%
\truex{1300}\truey{2000}\truez{200}%
\put(-\value{x},0){\oval(\value{y},#200)[l]}%
\put(-\value{x},0){\makebox(0,0){\begin{picture}(0,#200)%
\put(0,0){\line(1,0){\value{z}}}%
\put(0,#200){\vector(1,0){\value{z}}}\end{picture}}}%
\truex{2400}%
\put(-\value{x},0){\makebox(0,0)[r]{${#1}$}}%
\end{picture}}%

% West\North curved arrow of length [1]x100 units
\newcommand{\wncurve}[1]{\Wncurve{}{#1}}%




% FREE SLOPE ARROWS

% The following arrows have formal dimensions (0,0)

% arrow with name [1] at position [2,3],
% origin [4,5], slope [6,7] and horizontal extent [8]x100 units,
\newcommand{\Freear}[8]{\begin{picture}(0,0)%
\put(#400,#500){\vector(#6,#7){#800}}%
\truex{#200}\truey{#300}%
\put(\value{x},\value{y}){$#1$}\end{picture}}%

% arrow with origin [1,2], slope [3,4] and horizontal length [5]x100 units
\newcommand{\freear}[5]{\Freear{}{0}{0}{#1}{#2}{#3}{#4}{#5}}%



% INITIALIZATION

\newcount\SCALE%

\newcount\NUMBER%

\newcount\LINE%

\newcount\COLUMN%

\newcount\WIDTH%

\newcount\SOURCE%

\newcount\ARROW%

\newcount\TARGET%

\newcount\ARROWLENGTH%

\newcount\NUMBEROFDOTS%

\newcounter{x}%

\newcounter{y}%

\newcounter{z}%

\newcounter{horizontal}%

\newcounter{vertical}%

\newskip\itemlength%

\newskip\firstitem%

\newskip\seconditem%

\newcommand{\printarrow}{}%





% MACROS FOR DESIGNING DIAGRAMS


% \truex{n} divides nx100 by the scaling factor and puts the result
% in counter x
\newcommand{\truex}[1]{%
\NUMBER=#1%
\multiply\NUMBER by 100%
\divide\NUMBER by \SCALE%
\setcounter{x}{\NUMBER}}%

% \truey{n} divides nx100 by the scaling factor and puts the result
% in counter y
\newcommand{\truey}[1]{%
\NUMBER=#1%
\multiply\NUMBER by 100%
\divide\NUMBER by \SCALE%
\setcounter{y}{\NUMBER}}%

% \truez{n} divides nx100 by the scaling factor and puts the result
% in counter z
\newcommand{\truez}[1]{%
\NUMBER=#1%
\multiply\NUMBER by 100%
\divide\NUMBER by \SCALE%
\setcounter{z}{\NUMBER}}%


% \changecounters computes the values of the various parameters required
% to design an arrow with adjusted length.
\newcommand{\changecounters}[1]{%
\SOURCE=\ARROW%
\ARROW=\TARGET%
\settowidth{\itemlength}{#1}%
\ifdim \itemlength > 2800\unitlength%
\addtolength{\itemlength}{-2800\unitlength}%
\TARGET=\itemlength%
\divide\TARGET by 1310%
\multiply\TARGET by 100%
\divide\TARGET by \SCALE%
\else%
\TARGET=0%
\fi%
\ARROWLENGTH=5000%
\advance\ARROWLENGTH by -\SOURCE%
\advance\ARROWLENGTH by -\TARGET%
\advance\SOURCE by -\TARGET}%

% \initialize initializes the counters required to produce the diagram
% and defines the formal dimensions of the diagram to be (0,0).
\newcommand{\initialize}[1]{%
\LINE=0%
\COLUMN=0%
\WIDTH=0%
\ARROW=0%
\TARGET=0%
\changecounters{#1}%
\renewcommand{\printarrow}{#1}%
\begin{center}%
\vspace{10pt}%
\begin{picture}(0,0)}%

% \DIAG starts the construction of an unscaled diagram
\newcommand{\DIAG}[1]{%
\SCALE=100%
\setlength{\unitlength}{655sp}%
\initialize{\mbox{$#1$}}}%

% \DIAGV{n} starts the construction of a diagram scaled by a factor
% n percent, computes the scaled unit length and the unscaling factor.
\newcommand{\DIAGV}[2]{%
\SCALE=#1%
\setlength{\unitlength}{655sp}%
\multiply\unitlength by \SCALE%
\divide\unitlength by 100%
\initialize{\mbox{$#2$}}}%

% \n introduces the next item of the diagram, computes its parameters
% and prints the previous item.
\newcommand{\n}[1]{%
\changecounters{\mbox{$#1$}}%
\put(\COLUMN,\LINE){\makebox(0,0){\printarrow}}%
\thinlines%
\renewcommand{\printarrow}{\mbox{$#1$}}%
\advance\COLUMN by 4000}%

% \nn prints the  last item of a line and starts a new line.
\newcommand{\nn}[1]{%
\put(\COLUMN,\LINE){\makebox(0,0){\printarrow}}%
\thinlines%
\ifnum \WIDTH < \COLUMN%
\WIDTH=\COLUMN%
\else%
\fi%
\advance\LINE by -4000%
\COLUMN=0%
\ARROW=0%
\TARGET=0%
\changecounters{\mbox{$#1$}}%
\renewcommand{\printarrow}{\mbox{$#1$}}}%

% \conclude prints the last item of the diagram and sets the
% dimensions of the diagram;
\newcommand{\conclude}{%
\put(\COLUMN,\LINE){\makebox(0,0){\printarrow}}%
\thinlines%
\ifnum \WIDTH < \COLUMN%
\WIDTH=\COLUMN%
\else%
\fi%
\setcounter{horizontal}{\WIDTH}%
\setcounter{vertical}{-\LINE}%
\end{picture}}%

% \diag prints the last item of the diagram and takes care of the spacing
\newcommand{\diag}{%
\conclude%
\raisebox{0pt}[0pt][\value{vertical}\unitlength]{}%
\hspace*{\value{horizontal}\unitlength}%
\vspace{10pt}%
\end{center}%
\setlength{\unitlength}{1pt}}%

% \diagv{t}{l}{b} prints the last item of the diagram and
% adds a t points extra space at the top of the diagram
% adds a l points extra space at the left of the diagram
% adds a b points extra space at the bottom of the diagram
\newcommand{\diagv}[3]{%
\conclude%
\NUMBER=#1%
\rule{0pt}{\NUMBER pt}%
\hspace*{-#2pt}%
\raisebox{0pt}[0pt][\value{vertical}\unitlength]{}%
\hspace*{\value{horizontal}\unitlength}%6
\NUMBER=#3%
\advance\NUMBER by 10%
\vspace*{\NUMBER pt}%
\end{center}%
\setlength{\unitlength}{1pt}}%

% \N normalizes the height of a vertex
\newcommand{\N}[1]%
{\raisebox{0pt}[7pt][0pt]{$#1$}}%

% \movename{f}{n}{m} moves the name f of the arrow n points right
% and m points up.
\newcommand{\movename}[3]{%
\hspace{#2pt}%
\raisebox{#3pt}[5pt][2pt]{\raisebox{#3pt}{$#1$}}%
\hspace{-#2pt}}%

% \movearrow{\...}{n}{m} moves arrow \... n points right
% and m points up.
\newcommand{\movearrow}[3]{%
\makebox[0pt]{%
\hspace{#2pt}\hspace{#2pt}%
\raisebox{#3pt}[0pt][0pt]{\raisebox{#3pt}{$#1$}}}}%

% \movevertex{A}{n}{m} moves vertex A n points right and m points up
\newcommand{\movevertex}[3]{%
\mbox{\hspace{#2pt}%
\raisebox{#3pt}{\raisebox{#3pt}{$#1$}}%
\hspace{-#2pt}}}%

% \crosslength{P}{Q} computes the formal dimensions of the
% superposition of pictures P and Q
\newcommand{\crosslength}[2]{%
\settowidth{\firstitem}{#1}%
\settowidth{\seconditem}{#2}%
\ifdim\firstitem < \seconditem%
\itemlength=\seconditem%
\else%
\itemlength=\firstitem%
\fi%
\divide\itemlength by 2%
\hspace{\itemlength}}%

% \cross{P}{Q} superposes pictures P and Q
\def\cross#1#2{%
\crosslength{\mbox{$#1$}}{\mbox{$#2$}}%
\begin{picture}(0,0)%
\put(0,0){\makebox(0,0){$#1$}}%
\thinlines%
\put(0,0){\makebox(0,0){$#2$}}%
\thinlines%
\end{picture}%
\crosslength{\mbox{$#1$}}{\mbox{$#2$}}}%

% \B prints the next arrow in bold-face type
\newcommand{\B}{\thicklines}


% SPECIAL SYMBOLS

% adjoint symbol
\newcommand{\adj}{\begin{picture}(9,6)%
\put(1,3){\line(1,0){6}}\put(7,0){\line(0,1){6}}%
\end{picture}}%

% commutative diagram symbol
\newcommand{\com}{\begin{picture}(12,8)%
\put(6,4){\oval(8,8)[b]}\put(6,4){\oval(8,8)[r]}%
\put(6,8){\vector(-1,0){2}}\end{picture}}%

% natural transformation with names [1],[2],[3]
\newcommand{\Nat}[3]{\raisebox{-2pt}%
{\begin{picture}(34,15)%
\put(2,10){\vector(1,0){30}}%
\put(2,0){\vector(1,0){30}}%
\put(13,2){$\Downarrow$}%
\put(20,3){$\scriptstyle{#2}$}%
\put(4,11){$\scriptstyle{#1}$}%
\put(4,1){$\scriptstyle{#3}$}%
\end{picture}}}%

% natural transformation
\def\nat{\raisebox{-2pt}%
{\begin{picture}(34,10)%
\put(2,10){\vector(1,0){30}}%
\put(2,0){\vector(1,0){30}}%
\put(13,2){$\Downarrow$}%
\end{picture}}}%

% pair of natural transformations with names [1],[2],[3],[4],[5]
\newcommand{\Binat}[5]{\raisebox{-7.5pt}%
{\begin{picture}(34,25)%
\put(2,20){\vector(1,0){30}}%
\put(2,10){\vector(1,0){30}}%
\put(2,0){\vector(1,0){30}}%
\put(13,12){$\Downarrow$}%
\put(13,2){$\Downarrow$}%
\put(20,13){$\scriptstyle{#2}$}%
\put(20,3){$\scriptstyle{#4}$}%
\put(4,21){$\scriptstyle{#1}$}%
\put(4,11){$\scriptstyle{#3}$}%
\put(4,1){$\scriptstyle{#5}$}%
\end{picture}}}%

% pair of natural transformations
\newcommand{\binat}{\raisebox{-7.5pt}%
{\begin{picture}(34,20)%
\put(2,20){\vector(1,0){30}}%
\put(2,10){\vector(1,0){30}}%
\put(2,0){\vector(1,0){30}}%
\put(13,12){$\Downarrow$}%
\put(13,2){$\Downarrow$}%
\end{picture}}}%

